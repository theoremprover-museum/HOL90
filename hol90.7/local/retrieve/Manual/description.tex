\chapter{The retrieve Library}

\section{Introduction}

This document describes the facilities provided by the \ml{retrieve} library
for the \HOL\ system~\cite{HOLbook,ImplHOL}. In {\HOL}88 the library was
called \ml{trs}.

One of the most time-consuming activities when using the \HOL\ System,
particularly for people new to the system, is discovering the names of
built-in theorems required for a proof. Typically this involves manually
searching for the theorem in the documentation.

If the user can guess roughly what the name of the theorem is, then it may be
possible to use the operating system to search for strings within theory
files, and hence locate the theorem. It would be better to search for the
theorem by specifying its structure, since the user should have a fairly good
idea of what this is. The Theorem Retrieval System provides such searching by
structure. It does this from within a \HOL\ session.

The system uses \HOL\ terms as a means of specifying the structure of the
required theorem. The terms given are interpreted as templates or patterns
for the theorem. Variables in the template term can match subterms in the
theorem. The user may only know the form of some subterm of the theorem, so
it is not sufficient just to be able to match a template against the whole of
the conclusion of a theorem. The retrieval system allows searches for theorems
containing subterms which match a pattern. Logical connectives are also
available to build up more sophisticated search specifications.

It is not just the names of built-in theorems that may not be known. A user
may have forgotten the name under which he/she stored a personal theorem.
For this reason, the retrieval system allows user theories to be searched
as well as the built-in theories.


\section{Using the library}

The \ml{retrieve} library can be loaded into a \HOL\ session using
\ml{load\_library\_in\_place}\index{load\_library\_in\_place@{\ptt load\_library\_in\_place}}
(see the \HOL\ documentation for a general description of library loading).
This causes the \ML\ identifiers in the library to be loaded into \HOL. In
addition, the \HOL\ help\index{help!updating search path} search path is
updated with pathnames to online help files for the \ML\ identifiers in the
library. 

The following session shows how the \ml{retrieve} library may be loaded using
the function \ml{load\_library\_in\_place}:

\setcounter{sessioncount}{1}

\begin{session}\begin{verbatim}
- load_library_in_place (find_library "retrieve");

Loading the library "retrieve".

The library "HOL" is already loaded.

  The parent libraries of library "retrieve" have been loaded.
.
.
.
  The code of library "retrieve" has been loaded.

  The help path for library "retrieve" has been adjusted.

The library "retrieve" is loaded.
val it = () : unit
- 
\end{verbatim}\end{session}

\noindent
This causes a number of \ML\ signatures and structures to be defined. The most
important of these is \ml{Retrieve} which contains identifiers that are useful
to the user. Opening this structure will cause the identifiers to be made
available at the top level of the \ML\ session, and a few are given infix
status:

\begin{session}\begin{verbatim}
- open Retrieve;
open Retrieve
val BigAnd = fn : thmpattern list -> thmpattern
.
.
.
val List = fn : foundthm list -> source
- \end{verbatim}\end{session}

Below are some examples of the kind of searches that can be performed using
the \ml{retrieve} library.

Search for axioms in the theory {\small\verb%theory1%}:

\begin{boxed}\begin{verbatim}
- FT (kind Axiom)
=    (Paths [Theory "theory1"]);
\end{verbatim}\end{boxed}

Search for theorems whose name contains the string {\small\verb%"CONJ"%}.
{\small\verb%theory1%} and all its ancestors are searched, excluding
{\small\verb%theory2%} and its ancestors:

\begin{boxed}\begin{verbatim}
- FT (thmname "*CONJ*")
=    (Paths [Ancestors (["theory1"],["theory2"])]);
\end{verbatim}\end{boxed}

Search the theory {\small\verb%theory1%} and all of its ancestors for theorems
with a conclusion matching the pattern
{\small\verb%(--`!x y. x /\ y = y /\ x`--)%}. The hypotheses of the theorem
can be anything:

\begin{boxed}\begin{verbatim}
- FT (conc (--`!x y. x /\ y = y /\ x`--))
=    (Paths [Ancestors (["theory1"],[])]);
\end{verbatim}\end{boxed}

Search for theorems which contain a term of the form
{\small\verb%(--`a /\ b`--)%} (where {\small\verb%a%} and {\small\verb%b%} are
arbitrary Boolean-valued terms) somewhere within the structure of the
conclusion. This is done by matching any conclusion using the pattern
{\small\verb%(--`x:bool`--)%} and then performing a side-condition test on
{\small\verb%(--`x:bool`--)%} to see if it contains a term of the required
form. The theory {\small\verb%theory1%} is searched first, followed by
{\small\verb%theory2%} and all of its ancestors. Note that:

\begin{small}\begin{verbatim}
   (conc (--`x:bool`--)) Where ((--`x:bool`--) contains (--`a /\ b`--))
\end{verbatim}\end{small}

\noindent
can be abbreviated to:

\begin{small}\begin{verbatim}
   (--`conc:bool`--) contains (--`a /\ b`--)
\end{verbatim}\end{small}

\begin{boxed}\begin{verbatim}
- FT ((conc (--`x:bool`--)) Where ((--`x:bool`--) contains (--`a /\ b`--)))
=    (Paths [Theory "theory1",Ancestors (["theory2"],[])]);
\end{verbatim}\end{boxed}

Search the list of theorems \ml{fthml} for theorems with at least one
hypothesis:

\begin{boxed}\begin{verbatim}
- FT (hypP [(--`x:bool`--)])
=    (List fthml);
\end{verbatim}\end{boxed}

Search for theorems with exactly one hypothesis. The source for the search is
extracted from the search step \ml{s}, which was obtained from a previous
search:

\begin{boxed}\begin{verbatim}
- FT (hypF [(--`x:bool`--)])
=    (List_from s);
\end{verbatim}\end{boxed}

Search {\small\verb%theory1%} for theorems with exactly two hypotheses, both
of which are equations, and where the right-hand side of one of the equations
is either the number {\small\verb%1%} or a variable:

\begin{boxed}\begin{verbatim}
- FT ((hypF [(--`(a:'a) = b`--),(--`(c:'b) = d`--)])
=        Where (((--`b:'a`--) matches (--`1`--)) Orelse
=               (test1term is_var (--`b:'a`--))))
=    (Paths [Theory "theory1"]);
\end{verbatim}\end{boxed}

Search {\small\verb%theory1%} for theorems whose conclusions contain exactly
two conjunctions:

\begin{boxed}\begin{verbatim}
- FT (test1term (fn t => (cnt_conj t) = 2) (--`conc:bool`--))
=    (Paths [Theory "theory1"]);
\end{verbatim}\end{boxed}

\noindent
The function \ml{cnt\_conj} is defined by:

\begin{small}\begin{verbatim}
   fun cnt_conj t =
      let val n = if (is_conj t) then 1 else 0
      in  (n + (cnt_conj (body t))) handle HOL_ERR _ =>
          (n + (cnt_conj (rator t)) + (cnt_conj (rand t))) handle HOL_ERR _ =>
          n
      end;
\end{verbatim}\end{small}

\noindent
Given a term, it returns an integer. The value of the integer is the number of
conjunctions in the term.


\chapter{Example Sessions}

This chapter consists of several \HOL\ sessions. These are intended to provide
a tutorial introduction to the Theorem Retrieval System. At the beginning of
each session the Theorem Retrieval System is loaded into \HOL. The material
displayed within boxes is part of the session. The other text is a commentary
on what is happening within the session.


\input{simple}
\input{patterns}
\input{paths}
\input{sidecond}
\input{errors}


\chapter{New ML Types in the retrieve Library}

{\def\_{{\char'137}}                     % \tt style `_' character

\begin{center}
\begin{tabular}{|l|l|}
\hline
{\it ML type}                 & {\it Description}\\
\hline
{\small\tt thmkind}           & The `kind' of a theorem: axiom, definition or
derived theorem \\
{\small\tt foundthm}          & A theorem, its name, its theory segment and its
`kind' \\
{\small\tt thmpattern}        & Patterns for theorems \\
{\small\tt searchpath}        & Paths through theory hierarchies \\
{\small\tt source}            & Sources for a search: either a list of theorems
or a search path \\
{\small\tt searchstep}        & Steps within a search: one step for each theory
segment \\
\hline
\end{tabular}
\end{center}

}
