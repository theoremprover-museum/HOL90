\chapter*{Acknowledgements}\markboth{Acknowledgements}{Acknowledgements}

The three volumes \TUTORIAL, \DESCRIPTION\ and \REFERENCE\ were
produced at the Cambridge Research Center of SRI International with
the support of DSTO Australia.

The \HOL\ documentation project was managed by Mike Gordon, who also
wrote parts of \DESCRIPTION\ and \TUTORIAL\ using material based on an
early paper describing the \HOL\ system\footnote{M.J.C.\ Gordon, `HOL:
a Proof Generating System for Higher Order Logic', in: {\it VLSI
Specification, Verification and Synthesis\/}, edited by G.\ Birtwistle
and P.A.\ Subrahmanyam, (Kluwer Academic Publishers, 1988), pp.\
73--128.} and {\sl The ML Handbook\/}\footnote{{\sl The ML Handbook},
unpublished report from Inria by Guy Cousineau, Mike Gordon, G\'erard
Huet, Robin Milner, Larry Paulson and Chris Wadsworth.}.  Other
contributers to \DESCRIPTION\ incude Avra Cohn, who contributed
material on theorems, rules, conversions and tactics, and also
composed the index (which was typeset by Juanito Camilleri); Tom
Melham, who wrote the sections describing type definitions, the
concrete type package and the `resolution' tactics; and  Andy Pitts, who
devised the set-theoretic semantics of the \HOL\ logic and wrote the
material describing it.

The first edition of \TUTORIAL\ contained case studies on
microprocessor systems (by Jeff Joyce), protocol verification (by
Rachel Cardell-Oliver) and modular arithmetic based on group theory
(by Elsa Gunter).  These are now separate documents in the \HOL\
distribution directory {\small\verb%hol/Training/studies%}.  The
chapter in \TUTORIAL\ on the proof of the binomial theorem in \HOL{}
was written by Andy Gordon.

The second edition of \REFERENCE\ was a joint effort by the Cambridge
\HOL\ group.

The original document design used \LaTeX\ macros supplied by Elsa
Gunter, Tom Melham and Larry Paulson.  The typesetting of all three
volumes was managed by Tom Melham.  The conversion of the {\tt troff}
sources of {\sl The ML Handbook\/} to \LaTeX\ was done by Inder
Dhingra and John Van Tassel.  The cover design is by Arnold Smith, who
used a photograph of a `snow watching lantern' taken by Avra Cohn (in
whose garden the original object resides).  John Van Tassel composed
the \LaTeX\ picture of the lantern.


Many people other than those listed above have contributed to the \HOL\
documentation effort, either by providing material, or by sending lists of
errors in the first edition.  Thanks to everyone who helped, and thanks to DSTO
and SRI for their generous support.

