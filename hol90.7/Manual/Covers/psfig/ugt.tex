\documentstyle[12pt]{article}
\begin{document}
% psfigTeX macros
%
% All software, documentation, and related files in this distribution of
% psfig/tex are Copyright (c) 1987 Trevor J. Darrell
%
% Permission is granted for use and non-profit distribution of psfig/tex 
% providing that this notice be clearly maintained, but the right to
% distribute any portion of psfig/tex for profit or as part of any commercial
% product is specifically reserved for the author.
%
% Psfig/TeX Release 1.2
%
% file last modified: $Header: psfig.tex,v 1.3 88/11/22 01:23:34 van Exp $
%
\catcode`\@=11\relax
\newwrite\@unused
\def\typeout#1{{\let\protect\string\immediate\write\@unused{#1}}}
\typeout{psfig/tex 1.2}
%
% @psdo control structure -- similar to Latex @for.
% I redefined these with different names so that psfig can
% be used with TeX as well as LaTeX, and so that it will not 
% be vunerable to future changes in LaTeX's internal
% control structure,
%
\def\@nnil{\@nil}
\def\@empty{}
\def\@psdonoop#1\@@#2#3{}
\def\@psdo#1:=#2\do#3{\edef\@psdotmp{#2}\ifx\@psdotmp\@empty \else
    \expandafter\@psdoloop#2,\@nil,\@nil\@@#1{#3}\fi}
\def\@psdoloop#1,#2,#3\@@#4#5{\def#4{#1}\ifx #4\@nnil \else
       #5\def#4{#2}\ifx #4\@nnil \else#5\@ipsdoloop #3\@@#4{#5}\fi\fi}
\def\@ipsdoloop#1,#2\@@#3#4{\def#3{#1}\ifx #3\@nnil 
       \let\@nextwhile=\@psdonoop \else
      #4\relax\let\@nextwhile=\@ipsdoloop\fi\@nextwhile#2\@@#3{#4}}
\def\@tpsdo#1:=#2\do#3{\xdef\@psdotmp{#2}\ifx\@psdotmp\@empty \else
    \@tpsdoloop#2\@nil\@nil\@@#1{#3}\fi}
\def\@tpsdoloop#1#2\@@#3#4{\def#3{#1}\ifx #3\@nnil 
       \let\@nextwhile=\@psdonoop \else
      #4\relax\let\@nextwhile=\@tpsdoloop\fi\@nextwhile#2\@@#3{#4}}
% 
%
\def\psdraft{
	\def\@psdraft{0}
	%\typeout{draft level now is \@psdraft \space . }
}
\def\psfull{
	\def\@psdraft{100}
	%\typeout{draft level now is \@psdraft \space . }
}
\psfull
\newif\if@prologfile
\newif\if@postlogfile
\newif\if@noisy
\def\pssilent{
      \@noisyfalse
}
\def\psnoisy{
      \@noisytrue
}
\psnoisy
%%% These are for the option list.
%%% A specification of the form a = b maps to calling \@p@@sa{b}
\newif\if@bbllx
\newif\if@bblly
\newif\if@bburx
\newif\if@bbury
\newif\if@height
\newif\if@width
\newif\if@rheight
\newif\if@rwidth
\newif\if@clip
\newif\if@verbose
\def\@p@@sclip#1{\@cliptrue}
\def\@p@@sfile#1{%\typeout{file is #1}
		   \def\@p@sfile{#1}
}
\def\@p@@sfigure#1{\def\@p@sfile{#1}}
\def\@p@@sbbllx#1{
		%\typeout{bbllx is #1}
		\@bbllxtrue
		\dimen100=#1
		\edef\@p@sbbllx{\number\dimen100}
}
\def\@p@@sbblly#1{
		%\typeout{bblly is #1}
		\@bbllytrue
		\dimen100=#1
		\edef\@p@sbblly{\number\dimen100}
}
\def\@p@@sbburx#1{
		%\typeout{bburx is #1}
		\@bburxtrue
		\dimen100=#1
		\edef\@p@sbburx{\number\dimen100}
}
\def\@p@@sbbury#1{
		%\typeout{bbury is #1}
		\@bburytrue
		\dimen100=#1
		\edef\@p@sbbury{\number\dimen100}
}
\def\@p@@sheight#1{
		\@heighttrue
		\dimen100=#1
   		\edef\@p@sheight{\number\dimen100}
		%\typeout{Height is \@p@sheight}
}
\def\@p@@swidth#1{
		%\typeout{Width is #1}
		\@widthtrue
		\dimen100=#1
		\edef\@p@swidth{\number\dimen100}
}
\def\@p@@srheight#1{
		%\typeout{Reserved height is #1}
		\@rheighttrue
		\dimen100=#1
		\edef\@p@srheight{\number\dimen100}
}
\def\@p@@srwidth#1{
		%\typeout{Reserved width is #1}
		\@rwidthtrue
		\dimen100=#1
		\edef\@p@srwidth{\number\dimen100}
}
\def\@p@@ssilent#1{
	      \@verbosefalse
}
\def\@p@@sprolog#1{\@prologfiletrue\def\@prologfileval{#1}}
\def\@p@@spostlog#1{\@postlogfiletrue\def\@postlogfileval{#1}}
\def\@cs@name#1{\csname #1\endcsname}
\def\@setparms#1=#2,{\@cs@name{@p@@s#1}{#2}}
%
% initialize the defaults (size the size of the figure)
%
\def\ps@init@parms{
		\@bbllxfalse \@bbllyfalse
		\@bburxfalse \@bburyfalse
		\@heightfalse \@widthfalse
		\@rheightfalse \@rwidthfalse
		\def\@p@sbbllx{}\def\@p@sbblly{}
		\def\@p@sbburx{}\def\@p@sbbury{}
		\def\@p@sheight{}\def\@p@swidth{}
		\def\@p@srheight{}\def\@p@srwidth{}
		\def\@p@sfile{}
		\def\@p@scost{10}
		\def\@sc{}
		\@prologfilefalse
		\@postlogfilefalse
		\@clipfalse
		\if@noisy{
			\@verbosetrue
		}\else{
			\@verbosefalse
		}\fi
}
%
% Go through the options setting things up.
%
\def\parse@ps@parms#1{
	 	\@psdo\@psfiga:=#1\do
		   {\expandafter\@setparms\@psfiga,}}
%
% Compute bb height and width
%
\newif\ifno@bb
\newif\ifnot@eof
\newread\ps@stream
\def\bb@missing{
	\if@verbose{
		\typeout{psfig: searching \@p@sfile \space  for bounding box}
	}\fi
	\openin\ps@stream=\@p@sfile
	\no@bbtrue
	\not@eoftrue
	\catcode`\%=12
	\loop
		\read\ps@stream to \line@in
		\global\toks200=\expandafter{\line@in}
		\ifeof\ps@stream \not@eoffalse \fi
		%\typeout{ looking at :: \the\toks200 }
		\@bbtest{\toks200}
		\if@bbmatch\not@eoffalse\expandafter\bb@cull\the\toks200\fi
	\ifnot@eof \repeat
	\catcode`\%=14
}	
\catcode`\%=12
\newif\if@bbmatch
\def\@bbtest#1{\expandafter\@a@\the#1%%BoundingBox:\@bbtest\@a@}
\long\def\@a@#1%%BoundingBox:#2#3\@a@{\ifx\@bbtest#2\@bbmatchfalse\else\@bbmatchtrue\fi}
\long\def\bb@cull#1 #2 #3 #4 #5 {
	\dimen100=#2 bp\edef\@p@sbbllx{\number\dimen100}
	\dimen100=#3 bp\edef\@p@sbblly{\number\dimen100}
	\dimen100=#4 bp\edef\@p@sbburx{\number\dimen100}
	\dimen100=#5 bp\edef\@p@sbbury{\number\dimen100}
	\no@bbfalse
}
\catcode`\%=14
%
\def\compute@bb{
		\no@bbfalse
		\if@bbllx \else \no@bbtrue \fi
		\if@bblly \else \no@bbtrue \fi
		\if@bburx \else \no@bbtrue \fi
		\if@bbury \else \no@bbtrue \fi
		\ifno@bb \bb@missing \fi
		\ifno@bb \typeout{FATAL ERROR: no bb supplied or found}
			\no-bb-error
		\fi
		%
		\count203=\@p@sbburx
		\count204=\@p@sbbury
		\advance\count203 by -\@p@sbbllx
		\advance\count204 by -\@p@sbblly
		\edef\@bbw{\number\count203}
		\edef\@bbh{\number\count204}
		%\typeout{ bbh = \@bbh, bbw = \@bbw }
}
%
% \in@hundreds performs #1 * (#2 / #3) correct to the hundreds,
%	then leaves the result in @result
%
\def\in@hundreds#1#2#3{\count240=#2 \count241=#3
		     \count100=\count240	% 100 is first digit #2/#3
		     \divide\count100 by \count241
		     \count101=\count100
		     \multiply\count101 by \count241
		     \advance\count240 by -\count101
		     \multiply\count240 by 10
		     \count101=\count240	%101 is second digit of #2/#3
		     \divide\count101 by \count241
		     \count102=\count101
		     \multiply\count102 by \count241
		     \advance\count240 by -\count102
		     \multiply\count240 by 10
		     \count102=\count240	% 102 is the third digit
		     \divide\count102 by \count241
		     \count200=#1\count205=0
		     \count201=\count200
			\multiply\count201 by \count100
		 	\advance\count205 by \count201
		     \count201=\count200
			\divide\count201 by 10
			\multiply\count201 by \count101
			\advance\count205 by \count201
			%
		     \count201=\count200
			\divide\count201 by 100
			\multiply\count201 by \count102
			\advance\count205 by \count201
			%
		     \edef\@result{\number\count205}
}
\def\compute@wfromh{
		% computing : width = height * (bbw / bbh)
		\in@hundreds{\@p@sheight}{\@bbw}{\@bbh}
		%\typeout{ \@p@sheight * \@bbw / \@bbh, = \@result }
		\edef\@p@swidth{\@result}
		%\typeout{w from h: width is \@p@swidth}
}
\def\compute@hfromw{
		% computing : height = width * (bbh / bbw)
		\in@hundreds{\@p@swidth}{\@bbh}{\@bbw}
		%\typeout{ \@p@swidth * \@bbh / \@bbw = \@result }
		\edef\@p@sheight{\@result}
		%\typeout{h from w : height is \@p@sheight}
}
\def\compute@handw{
		\if@height 
			\if@width
			\else
				\compute@wfromh
			\fi
		\else 
			\if@width
				\compute@hfromw
			\else
				\edef\@p@sheight{\@bbh}
				\edef\@p@swidth{\@bbw}
			\fi
		\fi
}
\def\compute@resv{
		\if@rheight \else \edef\@p@srheight{\@p@sheight} \fi
		\if@rwidth \else \edef\@p@srwidth{\@p@swidth} \fi
}
%		
% Compute any missing values
\def\compute@sizes{
	\compute@bb
	\compute@handw
	\compute@resv
}
%
% \psfig
% usage : \psfig{file=, height=, width=, bbllx=, bblly=, bburx=, bbury=,
%			rheight=, rwidth=, clip=}
%
% "clip=" is a switch and takes no value, but the `=' must be preset.
\def\psfig#1{\vbox {
	% do a zero width hard space so that a single
	% \psfig in a centering enviornment will behave nicely
	%{\setbox0=\hbox{\ }\ \hskip-\wd0}
	%
	\ps@init@parms
	\parse@ps@parms{#1}
	\compute@sizes
	%
	\ifnum\@p@scost<\@psdraft{
		\if@verbose{
			\typeout{psfig: including \@p@sfile \space }
		}\fi
		%
		\special{ps::[begin] 	\@p@swidth \space \@p@sheight \space
				\@p@sbbllx \space \@p@sbblly \space
				\@p@sbburx \space \@p@sbbury \space
				startTexFig \space }
		\if@clip{
			\if@verbose{
				\typeout{(clip)}
			}\fi
			\special{ps:: doclip \space }
		}\fi
		\if@prologfile
		    \special{ps: plotfile \@prologfileval \space } \fi
		\special{ps: plotfile \@p@sfile \space }
		\if@postlogfile
		    \special{ps: plotfile \@postlogfileval \space } \fi
		\special{ps::[end] endTexFig \space }
		% Create the vbox to reserve the space for the figure
		\vbox to \@p@srheight true sp{
			\hbox to \@p@srwidth true sp{
				\hss
			}
		\vss
		}
	}\else{ % draft figure, just reserve the space and print the
		% path name.
		\hbox {%
			\vrule\kern-.4pt
			\vbox to \@p@srheight true sp{%
				\hrule
				\vfil
				\hbox to \@p@srwidth true sp{%
					\hss
					\@p@sfile
					\hss
				}%
				\vfil
				\hrule
			}%
			\vrule\kern-.4pt
		}%
	}\fi
}}
\catcode`\@=12\relax


\def\Ps{{\sc Post\-Script} }
%
\title{Psfig/\TeX\ 1.1 Users Guide}
\author{Trevor Darrell\\
trevor@grasp.cis.upenn.edu\\
Computer and Information Science\\
University of Pennsylvania}
\date{July 1987}
\maketitle
%
\section{Introduction}
The \TeX\ typesetting system is a powerful tool in the preparation of the 
written word, yet when the time comes to add figures or pictures to a document,
one traditionally had to resort to tedious manual paste-up. 
With the advent of the \Ps page description language,
which allows the `nesting' of environments and is rapidly becoming
a {\it de facto} standard, it is now possible to merge graphics
directly into a document. 
Psfig/\TeX\ is a macro package for \TeX\ that facilitates the
inclusion of arbitrary \Ps figures into \TeX\ documents.
Figures are automatically scaled and positioned
on the page, and the proper amount of space is reserved.
Some possible figures include:
\par
\hbox{
\hspace{.3in}
\vbox{\psfig{figure=figs/zip.ps}\vspace{.5in}}
\psfig{figure=figs/piechart.ps,height=1.5in}
\vbox{\psfig{figure=figs/starlines.ps}\vspace{.6in}}
}
\par
Custom characters such as
`\psfig{figure=figs/pretzel.ps,height=8pt,silent=}' and 
`\psfig{figure=figs/cm.ps,height=8pt,silent=}'
may be created and used freely throughout a document. Since the
Macintosh drawing applications produce \Ps, they can be used to create
figures.
%
\par
\section{Including a figure}
To include a \Ps figure with psfig, simply load the psfig macros at the 
beginning of your document with
\par
\begin{quote}
{\tt\verb+% psfigTeX macros
%
% All software, documentation, and related files in this distribution of
% psfig/tex are Copyright (c) 1987 Trevor J. Darrell
%
% Permission is granted for use and non-profit distribution of psfig/tex 
% providing that this notice be clearly maintained, but the right to
% distribute any portion of psfig/tex for profit or as part of any commercial
% product is specifically reserved for the author.
%
% Psfig/TeX Release 1.2
%
% file last modified: $Header: psfig.tex,v 1.3 88/11/22 01:23:34 van Exp $
%
\catcode`\@=11\relax
\newwrite\@unused
\def\typeout#1{{\let\protect\string\immediate\write\@unused{#1}}}
\typeout{psfig/tex 1.2}
%
% @psdo control structure -- similar to Latex @for.
% I redefined these with different names so that psfig can
% be used with TeX as well as LaTeX, and so that it will not 
% be vunerable to future changes in LaTeX's internal
% control structure,
%
\def\@nnil{\@nil}
\def\@empty{}
\def\@psdonoop#1\@@#2#3{}
\def\@psdo#1:=#2\do#3{\edef\@psdotmp{#2}\ifx\@psdotmp\@empty \else
    \expandafter\@psdoloop#2,\@nil,\@nil\@@#1{#3}\fi}
\def\@psdoloop#1,#2,#3\@@#4#5{\def#4{#1}\ifx #4\@nnil \else
       #5\def#4{#2}\ifx #4\@nnil \else#5\@ipsdoloop #3\@@#4{#5}\fi\fi}
\def\@ipsdoloop#1,#2\@@#3#4{\def#3{#1}\ifx #3\@nnil 
       \let\@nextwhile=\@psdonoop \else
      #4\relax\let\@nextwhile=\@ipsdoloop\fi\@nextwhile#2\@@#3{#4}}
\def\@tpsdo#1:=#2\do#3{\xdef\@psdotmp{#2}\ifx\@psdotmp\@empty \else
    \@tpsdoloop#2\@nil\@nil\@@#1{#3}\fi}
\def\@tpsdoloop#1#2\@@#3#4{\def#3{#1}\ifx #3\@nnil 
       \let\@nextwhile=\@psdonoop \else
      #4\relax\let\@nextwhile=\@tpsdoloop\fi\@nextwhile#2\@@#3{#4}}
% 
%
\def\psdraft{
	\def\@psdraft{0}
	%\typeout{draft level now is \@psdraft \space . }
}
\def\psfull{
	\def\@psdraft{100}
	%\typeout{draft level now is \@psdraft \space . }
}
\psfull
\newif\if@prologfile
\newif\if@postlogfile
\newif\if@noisy
\def\pssilent{
      \@noisyfalse
}
\def\psnoisy{
      \@noisytrue
}
\psnoisy
%%% These are for the option list.
%%% A specification of the form a = b maps to calling \@p@@sa{b}
\newif\if@bbllx
\newif\if@bblly
\newif\if@bburx
\newif\if@bbury
\newif\if@height
\newif\if@width
\newif\if@rheight
\newif\if@rwidth
\newif\if@clip
\newif\if@verbose
\def\@p@@sclip#1{\@cliptrue}
\def\@p@@sfile#1{%\typeout{file is #1}
		   \def\@p@sfile{#1}
}
\def\@p@@sfigure#1{\def\@p@sfile{#1}}
\def\@p@@sbbllx#1{
		%\typeout{bbllx is #1}
		\@bbllxtrue
		\dimen100=#1
		\edef\@p@sbbllx{\number\dimen100}
}
\def\@p@@sbblly#1{
		%\typeout{bblly is #1}
		\@bbllytrue
		\dimen100=#1
		\edef\@p@sbblly{\number\dimen100}
}
\def\@p@@sbburx#1{
		%\typeout{bburx is #1}
		\@bburxtrue
		\dimen100=#1
		\edef\@p@sbburx{\number\dimen100}
}
\def\@p@@sbbury#1{
		%\typeout{bbury is #1}
		\@bburytrue
		\dimen100=#1
		\edef\@p@sbbury{\number\dimen100}
}
\def\@p@@sheight#1{
		\@heighttrue
		\dimen100=#1
   		\edef\@p@sheight{\number\dimen100}
		%\typeout{Height is \@p@sheight}
}
\def\@p@@swidth#1{
		%\typeout{Width is #1}
		\@widthtrue
		\dimen100=#1
		\edef\@p@swidth{\number\dimen100}
}
\def\@p@@srheight#1{
		%\typeout{Reserved height is #1}
		\@rheighttrue
		\dimen100=#1
		\edef\@p@srheight{\number\dimen100}
}
\def\@p@@srwidth#1{
		%\typeout{Reserved width is #1}
		\@rwidthtrue
		\dimen100=#1
		\edef\@p@srwidth{\number\dimen100}
}
\def\@p@@ssilent#1{
	      \@verbosefalse
}
\def\@p@@sprolog#1{\@prologfiletrue\def\@prologfileval{#1}}
\def\@p@@spostlog#1{\@postlogfiletrue\def\@postlogfileval{#1}}
\def\@cs@name#1{\csname #1\endcsname}
\def\@setparms#1=#2,{\@cs@name{@p@@s#1}{#2}}
%
% initialize the defaults (size the size of the figure)
%
\def\ps@init@parms{
		\@bbllxfalse \@bbllyfalse
		\@bburxfalse \@bburyfalse
		\@heightfalse \@widthfalse
		\@rheightfalse \@rwidthfalse
		\def\@p@sbbllx{}\def\@p@sbblly{}
		\def\@p@sbburx{}\def\@p@sbbury{}
		\def\@p@sheight{}\def\@p@swidth{}
		\def\@p@srheight{}\def\@p@srwidth{}
		\def\@p@sfile{}
		\def\@p@scost{10}
		\def\@sc{}
		\@prologfilefalse
		\@postlogfilefalse
		\@clipfalse
		\if@noisy{
			\@verbosetrue
		}\else{
			\@verbosefalse
		}\fi
}
%
% Go through the options setting things up.
%
\def\parse@ps@parms#1{
	 	\@psdo\@psfiga:=#1\do
		   {\expandafter\@setparms\@psfiga,}}
%
% Compute bb height and width
%
\newif\ifno@bb
\newif\ifnot@eof
\newread\ps@stream
\def\bb@missing{
	\if@verbose{
		\typeout{psfig: searching \@p@sfile \space  for bounding box}
	}\fi
	\openin\ps@stream=\@p@sfile
	\no@bbtrue
	\not@eoftrue
	\catcode`\%=12
	\loop
		\read\ps@stream to \line@in
		\global\toks200=\expandafter{\line@in}
		\ifeof\ps@stream \not@eoffalse \fi
		%\typeout{ looking at :: \the\toks200 }
		\@bbtest{\toks200}
		\if@bbmatch\not@eoffalse\expandafter\bb@cull\the\toks200\fi
	\ifnot@eof \repeat
	\catcode`\%=14
}	
\catcode`\%=12
\newif\if@bbmatch
\def\@bbtest#1{\expandafter\@a@\the#1%%BoundingBox:\@bbtest\@a@}
\long\def\@a@#1%%BoundingBox:#2#3\@a@{\ifx\@bbtest#2\@bbmatchfalse\else\@bbmatchtrue\fi}
\long\def\bb@cull#1 #2 #3 #4 #5 {
	\dimen100=#2 bp\edef\@p@sbbllx{\number\dimen100}
	\dimen100=#3 bp\edef\@p@sbblly{\number\dimen100}
	\dimen100=#4 bp\edef\@p@sbburx{\number\dimen100}
	\dimen100=#5 bp\edef\@p@sbbury{\number\dimen100}
	\no@bbfalse
}
\catcode`\%=14
%
\def\compute@bb{
		\no@bbfalse
		\if@bbllx \else \no@bbtrue \fi
		\if@bblly \else \no@bbtrue \fi
		\if@bburx \else \no@bbtrue \fi
		\if@bbury \else \no@bbtrue \fi
		\ifno@bb \bb@missing \fi
		\ifno@bb \typeout{FATAL ERROR: no bb supplied or found}
			\no-bb-error
		\fi
		%
		\count203=\@p@sbburx
		\count204=\@p@sbbury
		\advance\count203 by -\@p@sbbllx
		\advance\count204 by -\@p@sbblly
		\edef\@bbw{\number\count203}
		\edef\@bbh{\number\count204}
		%\typeout{ bbh = \@bbh, bbw = \@bbw }
}
%
% \in@hundreds performs #1 * (#2 / #3) correct to the hundreds,
%	then leaves the result in @result
%
\def\in@hundreds#1#2#3{\count240=#2 \count241=#3
		     \count100=\count240	% 100 is first digit #2/#3
		     \divide\count100 by \count241
		     \count101=\count100
		     \multiply\count101 by \count241
		     \advance\count240 by -\count101
		     \multiply\count240 by 10
		     \count101=\count240	%101 is second digit of #2/#3
		     \divide\count101 by \count241
		     \count102=\count101
		     \multiply\count102 by \count241
		     \advance\count240 by -\count102
		     \multiply\count240 by 10
		     \count102=\count240	% 102 is the third digit
		     \divide\count102 by \count241
		     \count200=#1\count205=0
		     \count201=\count200
			\multiply\count201 by \count100
		 	\advance\count205 by \count201
		     \count201=\count200
			\divide\count201 by 10
			\multiply\count201 by \count101
			\advance\count205 by \count201
			%
		     \count201=\count200
			\divide\count201 by 100
			\multiply\count201 by \count102
			\advance\count205 by \count201
			%
		     \edef\@result{\number\count205}
}
\def\compute@wfromh{
		% computing : width = height * (bbw / bbh)
		\in@hundreds{\@p@sheight}{\@bbw}{\@bbh}
		%\typeout{ \@p@sheight * \@bbw / \@bbh, = \@result }
		\edef\@p@swidth{\@result}
		%\typeout{w from h: width is \@p@swidth}
}
\def\compute@hfromw{
		% computing : height = width * (bbh / bbw)
		\in@hundreds{\@p@swidth}{\@bbh}{\@bbw}
		%\typeout{ \@p@swidth * \@bbh / \@bbw = \@result }
		\edef\@p@sheight{\@result}
		%\typeout{h from w : height is \@p@sheight}
}
\def\compute@handw{
		\if@height 
			\if@width
			\else
				\compute@wfromh
			\fi
		\else 
			\if@width
				\compute@hfromw
			\else
				\edef\@p@sheight{\@bbh}
				\edef\@p@swidth{\@bbw}
			\fi
		\fi
}
\def\compute@resv{
		\if@rheight \else \edef\@p@srheight{\@p@sheight} \fi
		\if@rwidth \else \edef\@p@srwidth{\@p@swidth} \fi
}
%		
% Compute any missing values
\def\compute@sizes{
	\compute@bb
	\compute@handw
	\compute@resv
}
%
% \psfig
% usage : \psfig{file=, height=, width=, bbllx=, bblly=, bburx=, bbury=,
%			rheight=, rwidth=, clip=}
%
% "clip=" is a switch and takes no value, but the `=' must be preset.
\def\psfig#1{\vbox {
	% do a zero width hard space so that a single
	% \psfig in a centering enviornment will behave nicely
	%{\setbox0=\hbox{\ }\ \hskip-\wd0}
	%
	\ps@init@parms
	\parse@ps@parms{#1}
	\compute@sizes
	%
	\ifnum\@p@scost<\@psdraft{
		\if@verbose{
			\typeout{psfig: including \@p@sfile \space }
		}\fi
		%
		\special{ps::[begin] 	\@p@swidth \space \@p@sheight \space
				\@p@sbbllx \space \@p@sbblly \space
				\@p@sbburx \space \@p@sbbury \space
				startTexFig \space }
		\if@clip{
			\if@verbose{
				\typeout{(clip)}
			}\fi
			\special{ps:: doclip \space }
		}\fi
		\if@prologfile
		    \special{ps: plotfile \@prologfileval \space } \fi
		\special{ps: plotfile \@p@sfile \space }
		\if@postlogfile
		    \special{ps: plotfile \@postlogfileval \space } \fi
		\special{ps::[end] endTexFig \space }
		% Create the vbox to reserve the space for the figure
		\vbox to \@p@srheight true sp{
			\hbox to \@p@srwidth true sp{
				\hss
			}
		\vss
		}
	}\else{ % draft figure, just reserve the space and print the
		% path name.
		\hbox {%
			\vrule\kern-.4pt
			\vbox to \@p@srheight true sp{%
				\hrule
				\vfil
				\hbox to \@p@srwidth true sp{%
					\hss
					\@p@sfile
					\hss
				}%
				\vfil
				\hrule
			}%
			\vrule\kern-.4pt
		}%
	}\fi
}}
\catcode`\@=12\relax

+}
\end{quote}
\par
then invoke the macro
\par
\begin{quote}
{\tt\verb+\+psfig\{figure={\it input}\}}
\end{quote}
\par
where {\it input} is the name of a \Ps file. 
psfig will automatically position the figure at the current place on the page, 
and reserve the proper amount of space in \TeX\ so that it doesn't conflict
with any other objects.
\par
For example, if we have a file called `piechart.ps' which contains the
\Ps code to draw the chart in the introduction, and it was the first figure
in our non-page reversed document, we would use the commands
\begin{quote}\small
{\tt\verb+\+input psfig } \\
{\tt\verb+\+centerline\{\verb+\+psfig\{figure=piechart.ps\}\}}
\end{quote}
Since no mention of size was made in the above example, psfig draws the figure 
at its natural size (as if it was printed directly on 
a \Ps printer.) The pie's natural size is several inches across, which
is a little large; the pie in the introduction was produced with:
\par
\begin{quote}\small
\tt\verb+\+centerline\{\verb+\+psfig\{figure=piechart.ps,height=1.5in\}\}
\end{quote}
\par
The {\tt height} option specifies how tall the figure should be on the
page. Since no {\tt width} was specified, the figure was scaled equally
in both dimensions.
By listing both a {\tt height} and a {\tt width}, figures can be scaled 
disproportionatly, with interesting results.
\par
For example:
\par
\centerline{\hbox{
\psfig{figure=figs/rosette.ps,height=.8in,width=.15in}
\psfig{figure=figs/rosette.ps,height=.8in,width=.35in}
\psfig{figure=figs/rosette.ps,height=.8in}
\psfig{figure=figs/rosette.ps,height=.8in,width=1.2in}
\psfig{figure=figs/rosette.ps,height=.8in,width=1.5in}
}}
\par
was produced with:
\begin{quote}\small
\tt\verb+\+centerline\{\verb+\+hbox\{ \\
\verb+\+psfig\{figure=rosette.ps,height=.8in,width=.15in\}\\
\verb+\+psfig\{figure=rosette.ps,height=.8in,width=.35in\}\\
\verb+\+psfig\{figure=rosette.ps,height=.8in\} \\
\verb+\+psfig\{figure=rosette.ps,height=.8in,width=1.2in\}\\
\verb+\+psfig\{figure=rosette.ps,height=.8in,width=1.5in\}\\
\}\} 
\end{quote}
\par
\subsection{Caveats}
For psfig to find the natural size of a figure, the figure must have a
proper bounding box comment; see section 5 below. Draft mode (also 
detailed below) should be used liberally to speed throughput. 
Also, some versions
of \LaTeX\ will fail to properly center a lone figure in a centering 
environment;
a good work-around is to precede the figure with a hard space. On very large documents with many figures, the printer memory
allocated to dvips may have to be limited. 
\Ps files produced with ps4014 need to have the clipping boundry set,
see section 6.3 below.
Finally, the
\verb+\psfig+ macro will be confused by extra white space or newlines in its
argument.
\par
\section{Generating PostScript}
Before you can include a figure, however, you must create
the \Ps file that describes it.
The most common methods for creating a \Ps figure are to 
use either a drawing application such as MacDraw, an image-to-ps or
textronix-to-ps translator, or to 
directly code \Ps by hand.
\begin{figure}
\centerline{\psfig{figure=figs/pathtext.ps,height=2in}}
\centerline{Figure 1. a \Ps figure}
\end{figure}
\par
\subsection{Macintosh files}
Using a Macintosh (or any other system that supports mouse-based drawing
utilities that use \Ps) is one of the easiest ways of creating
a figure (figure 2a.)
MacDraw is the recommended tool, since
it produces code that is independent of scaling (as opposed to MacPaint,
which produces a bitmap of the figure.)
There are several known methods of capturing a MacDraw/MacWrite figure
in \Ps form and transferring to the \TeX\ host; most involve some
mucking about with tricky control sequences, one is detailed in the appendix.
\par 
MacDraw creates a output file in the form of `QuickDraw' calls, which are 
interpreted as a set of \Ps procedures. These procedures are defined
in what is called the `macintosh prolog', which must be prepended 
to any macintosh file
to be sent to the printer.
There is a {\tt prolog} option in the psfig  macro to specify a file that
should be prepended to the figure. The name of the prolog is, of course,
site dependent; we have used `/usr/lib/ps/mac.pro'.
For example, if you had a file `frog.mac' that contained the macintosh
code to draw kermit (figure 2b), he could be included with:
\begin{quote}\small
{\tt\verb+\+psfig\{figure=frog.mac,prolog=mac.pro\}\}}
\end{quote}
\begin{figure}
\centerline{
\vbox{\vss\psfig{figure=figs/lab.ps,prolog=figs/mac.pro,width=2.5in}\vss}
\hss\vbox{\vss\psfig{figure=figs/frogfill.mac,prolog=figs/mac.pro,width=2in}\vss}}
\par
\centerline{Figure 2a \& 2b: Macintosh figures.}
\end{figure}
If there are many such figures, it is probable that the repeated inclusion
of the mac.pro file will cause a significant increase in file size and
transmission time.
A better method is to load the mac.pro file once, before the entire
document (using {\tt psprint -m}, on our systems).
\par
\subsection{Images (ph), plot, and other sources}
Any program that produces \Ps as output can be used for psfig figures.
For instance, the {\it ph} program will convert a bitmap image
to \Ps form and thus can be
used to include an image in a document (figure 3.)
\par
There are similar utilities that can convert files from unix plot(5)
or Textronix 4014 format into \Ps.
\begin{figure}
\centerline{
\vbox{\vss\psfig{figure=figs/trevor.ps,height=2in}\vss}
\hspace*{0.2in}{\psdraft\psfig{figure=figs/trevor.ps,height=2in}}
}
\par
\centerline{Figures 3 and 4: The author's image and draft equivalent}
\end{figure}
\par
\section{Draft figures}
Certain \Ps figures (such as large bitmap images being transmitted at
9600 baud) can tie up a slower \Ps device such as an
Apple LaserWriter for quite some time.
To circumvent this, a figure
may be printed in draft mode, which will occupy the same space on the
page as far as
\TeX\ is concerened, but it will just print the name of the file from 
which the figure is derived, and will not actually include it (figure 4.)
The macro {\tt \verb+\+psdraft} will switch into draft mode, and all
subsequent psfig macros will produce draft figures. The macro {\tt
\verb+\+psfull} will switch out of draft mode. 
\section{Bounding boxes}
To properly translate and scale a figure psfig must know its `natural'
position on the page; this information is present
in what is called the {\it bounding box} of a \Ps program. The 
bounding box is a outer limit to the marks created by a program,
and is specified as four coordinates of a rectangle: the lower-left $x$ coordinate
(bbllx), the lower-left $y$ coordinate (bblly), the upper-right
$x$ coordinate (bburx), and the upper-right $y$ coordinate (bbury).
Adobe has defined a convention whereby the bounding box of a program
is contained in a `bounding box comment'.
\footnote[1]{See `Appendix J: \Ps File Structuring Conventions' in
{\it The \Ps Language Reference Manual}}
This comment, which must be present in any file to be used as a psfig figure,
is a line of the form
\begin{quote}\small
\tt \verb+%%+BoundingBox: \it bbllx bblly bburx bbury
\end{quote}
All values are in \Ps points, relative to the {\it default}
transformation matrix. The only mandatory \Ps convention is
that the first line of the file should begin with the characters
`\verb+%+!' (a `\verb+%+' begins a comment in \Ps.) A good place for the
bounding box comment is as the second line of the file.
\par
\section{Advanced topics}
Casual users can safely skip this material unless they have run into 
one of the issues in section 2.1
\subsection{Internal structure}
In including a figure, the {\tt \verb+\psfig+} macro performs the following
operations:
First, if bounding box information (see below) is omitted from the list
of arguments, the
file containing the figure
is searched and the information recovered from the bounding box
comment. Then,
if both {\it height} and {\it width} are missing they are computed to
be the height and width of the bounding box. If only one is missing,
it is set to be the appropriate value such that there is no distorted
scaling.
If {\it rheight} or {\it rwidth} (see below) is missing it is presumed to
be the same as the height and width.
\par
The {\tt\verb+\+psfig\{figure={\it input}\}} macro
uses a vbox in \TeX\ to reserve the space for a figure.
The transformation of the postscript environment needed to
cause the figure to be printed at it's desired location is 
done using \Ps functions called with 
{\tt \verb+\+special}, as is the literal inclustion of the
text of the figure.
\footnote[2]{For more deatiled information about \Ps issues in the
interals of psfig, see {\it Psfig -- A Ditroff Preprocessor for
PostScript Figures} in the USENIX 87 proceedings, or {\it Bringing
troff up to speed} in the July 1987 issue of Unix Revew.}
\par
\subsection{Reserved size}
\psfig{figure=figs/box.ps,rheight=0bp,rwidth=0bp,height=1.25in,width=\textwidth,silent=}
\par
There are two sizes associated with each psfig figure: the size
at which it is to be printed on the page
and the size it reserves in \TeX. This latter size is appropriately
termed the {\it reserved size}, and is expressed as clauses of the form
`{\small\tt rheight={\it dimen}}'
and `{\tt rwidth={\it dimen}}'. If ommited, the reserved size defaults
to the real size. Some special effects need to be transparent
to \TeX\ and thus have a zero reserved size, such as the grey
box enclosing
this paragraph.
\par
\subsection{Clipping}
Normally a \Ps program can be expected to not mark the page 
outside its bounding box. If this is not the case, or if you
want to use the bounding box to isolate part of a larger figure,
there is an option that sets the \Ps clip path so that
no marks will show up outside the declared bounding box. Currently
this is invoked by adding a clause of the form `{\tt clip=}'.
Here a slice has been taken out of the pie chart in the example by
specifying a smaller bounding box and adding the clip option.
\par
\centerline{\psfig{figure=figs/piechart.ps,height=2in,bbllx=306bp,bblly=396bp,bburx=486bp,bbury=546bp,clip=}}
\par
Some \Ps programs use the clipping path to position their output on
the page; if a figure is being drawn at its natural size and position
despite psfig commands to the contrary, it may need the clip option.
\par
\subsection{\Ps environment}
The \Ps environment within psfig is fairly robust. All of the
usual \Ps operators will operate as desired; in particular 
the operators `showpage', `initgraphics', and `defaultmatrix' will
all behave consistently inside a figure, except that `showpage' will
only do an `initgraphics' and will {\it not} print or erase the current
page.
\par 
It is very possible, however, for a \Ps program to bypass the psfig
environment and disrupt a document. 
\par
\section{Acknowledgements}
\par
Ned Batchelder co-developed the original {\it troff} version of this
program with the author, and was responsible for much of the overall design.
Greg Hager provided an initial \TeX\ implementation.
Figure 1 and the three broken out figures in 
the introduction were taken from examples in {\it The PostScript Language
Tutorial and Cookbook.} 
\par 
\appendix
\newpage
\section*{Appendix A: Capturing \Ps files from a Macintosh -- one kludge}
In general, a \Ps file can be transferred from a Macintosh to another host
using any of the popular terminal emulators and a serial line. We have used
MacTerminal and Kermit without any problems.
\par
Slightly trickier is getting the \Ps into a file on the Macintosh. For
MacDraw and MacWord (and perhaps others), there is an undocumented
`feature' whereby the \Ps code can be diverted into a file rather than
being sent to a printer. Immediately after clicking `ok' from the print
menu, hit clover--F; the code will be placed in a file with the name 
`PostScript' (there is no known way to change this). Clover-K will 
capture the file {\it and} the lengthy prolog mentioned above.
\newpage
\section*{Appendix B: Installation}
\par
This version of psfig/tex was written to use certain {\tt \verb+\+special}
calls implemented in ArborText's {\it dvips}. It should be relatively
simple to port psfig to any dvi-to-ps program that supports file inclusion.
\par
The first step to install psfig is to print out a draft copy of this
guide simply by running {\tt latex ug.tex},
which presumably you have already done. 
\subsection*{B.1 Add the psfig \Ps functions to the dvips prolog}
Find the file called {\tt figtex.pro}, and {\it append} 
it to the {\tt dvips.pro}
file that is included every time a dvi file is converted to \Ps.
\subsection*{B.2 Print a full copy of this document}
Comment out the first line that contains {\tt \verb+\+psdraft } from the file
{\tt ug.tex}, then run {\tt latex ug.tex}
\subsection*{B.3 Copy psfig.tex to a well known location}
Preferably in the \TeX\ search path, so everyone can find it.
\end{document}

