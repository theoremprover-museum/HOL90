% define page layout
%\setlength{\textwidth}{5.7in}
%\setlength{\textheight}{8.5in}
%\setlength{\topmargin}{-0.125in}
%\setlength{\oddsidemargin}{18pt}
%\setlength{\evensidemargin}{18pt}
%\setlength{\columnseprule}{.4pt}
%\setlength{\headheight}{19pt}
%\setlength{\headsep}{18pt}
%\setlength{\footheight}{16pt}
%\setlength{\footskip}{34pt}
%\setlength{\headrulewidth}{2pt}
%\setlength{\footrulewidth}{0pt}

\renewcommand{\sectionmark}[1]{ }
\renewcommand{\subsectionmark}[1]{ }

\def\HDRULE{\rule[-.6\baselineskip]{\textwidth}{1pt}}
\newsavebox{\HDRULEBOX}
\savebox{\HDRULEBOX}[0pt]{\HDRULE}

%\lhead[]{{\sc Modelling Bit Vectors}}
%\chead[]{\usebox{\HDRULEBOX}}
%\rhead[]{{\sl\thepage\hspace{.5em}}}
%\cfoot[]{}


\input holmacs
%% FILE: tokmac.tex
%% AUTHOR: Wai WONG
%% DATE: 26 April 1993
%%
%% Copyright (C) 1992, 1993 Wai Wong
%%
%% This program is free software; you can redistribute it and/or modify
%% it under the terms of the GNU General Public License as published by
%% the Free Software Foundation; either version 2, or (at your option)
%% any later version.
%%
%% This program is distributed in the hope that it will be useful,
%% but WITHOUT ANY WARRANTY; without even the implied warranty of
%% MERCHANTABILITY or FITNESS FOR A PARTICULAR PURPOSE.  See the
%% GNU General Public License for more details.
%%
%% You should have received a copy of the GNU General Public License
%% along with this program; if not, write to the Free Software
%% Foundation, Inc., 675 Mass Ave, Cambridge, MA 02139, USA.  */
%%
%%
%% token name macros
%% 
%% Token name macros provide a convenient way of formatting and indexing
%% tokens. For example, if a class of tokens, such as identifiers in a
%% programming language, need to be typeset in a
%% special font and automatically indexed, one can define a macro, say
%% \TOKEN for this purpose. When this macro is used,
%% e.g., \TOKEN{foo}, `foo' will be set in a pre-defined font and an
%% index entry is generated automatically. This definition can be
%% automated. The macro
%%	\newtokmac{<name>}{<font>}
%% defines a new token macro whose name is <name>. This new macro
%% takes a single argument. It will be type set in the font specified
%% by <font>. An index entry is generated for every use of this
%% macro, and it will be set in the same font. The automatic
%% generation of index can be disabled by the command \indexfalse. It
%% can also be re-enable by the command \indextrue. A star variant of the
%% new macro is defined at the same time, i.e., \<name>* can be used.
%% When the satr variant is use, no index is generated no matter what
%% is the current setting of the index generation feature. The
%% argument of the token name macro can contain any ordinary character
%% plus underline(_).

% check to see if this macro file has already been loaded
\ifx\undefined\tokmac\def\tokmac{}\else\endinput\fi

\makeatletter
%% 
%% The macro \makeulother changes the catcode of the underline
%% character. It is called just before calling \dotoken or
%% \dotokenidx. It can be redefined to do other things which may
%% affect how TeX processes the token word.

\def\makeulother{\catcode`\_=12\relax}
%%
%% The token name macro  is implemented the macros: \dotoken,
%% \dotokenidx and \idxname. \dotaken is called when no index is
%% generated, while \dotokenidx is called when index is required. When
%% these macros are called, TeX is in a group in which the catcode of
%% the underline character has been changed to `other', but the actual
%% argument to the token name macro has not been read. These two
%% macros take two arguments: the first if a command for changing
%% font, and the second is the token word. the macro is called when
%% the index is processed.

\def\dotoken#1#2{\mbox{#1#2}\endgroup}
\def\idxname#1{\begingroup\makeulother\dotoken#1}
\def\dotokenidx#1#2{\mbox{#1#2}\index{#2@\string\idxname{#1}{#2}}\endgroup}
%%
%% The automatic generation of index is controlled by the conditional
%% \ifindex. The default is TRUE, i.e., the feature is enabled.
%%
\newif\ifindex \indextrue
%%
%% 
\def\@nametok{\begingroup\makeulother\dotoken}
\def\@nametokidx{\begingroup\makeulother\dotokenidx}
\def\@nametokidxtest{%
 \ifindex\let\doidx=\@nametokidx\else\let\doidx=\@nametok\fi\doidx}

\def\newtokmac#1#2{%
 \expandafter\def\csname#1\endcsname{\@ifstar{\@nametok{#2}}{\@nametokidxtest{#2}}}}

\makeatother


\newtokmac{mlname}{\tt}
\newtokmac{CONST}{\constfont}
\newtokmac{KEYWD}{\keyfont}
\newtokmac{para}{\tt}
\newtokmac{cmd}{\constfont}

\makeatletter
%\def\verb{\begingroup \catcode``=13 \@noligs
%\verbatim@font \let\do\@makeother \dospecials
%\@ifstar{\@sverb}{\@verb}}
%
%% Definitions of \@sverb and \@verb changed so \verb+ foo+  does not lose
%% leading blanks when it comes at the beginning of a line.
%% Change made 24 May 89. Suggested by Frank Mittelbach and Rainer Sch\"opf.
%%
%\def\@sverb#1{\def\@tempa ##1#1{\leavevmode\null##1\endgroup}\@tempa}
%
%\def\@verb{\@vobeyspaces \frenchspacing \@sverb}
%
\def\wordn{\verb|:word|$n$}
\def\word{\@ifnextchar[{\@word}{\@word[*]}}
\def\@word[#1]{\verb|:(#1)word|}
\def\sect{\@startsection {subsection}{1}{\z@}{-3.5ex plus -1ex minus
 -.2ex}{1.5ex plus .2ex}{\normalsize\bf}}
\def\subsect{\@startsection {subsubsection}{2}{\z@}{-3.5ex plus -1ex minus
 -.2ex}{-1em}{\footnotesize\bf}}
\def\inputmlfile#1{\begingroup \footnotesize \input#1 \endgroup}

%\renewenvironment{theindex}{\begin{multicols}{2}[\section*{\indexname}]%
% \columnseprule \z@ \columnsep 35\p@
% \parindent\z@ \parskip\z@ plus.3\p@\relax\let\item\@idxitem}{\end{multicols}}
%
%\def\@idxitem{\par\hangindent 40\p@}
%
%\def\subitem{\par\hangindent 40\p@ \hspace*{20\p@}}
%
%\def\subsubitem{\par\hangindent 40\p@ \hspace*{30\p@}}
%
%\def\indexspace{\par \vskip 10\p@ plus5\p@ minus3\p@\relax}
%
\makeatother


\def\NBWORD#1#2{\CONST{NBWORD}\,\CONST{#1}\,\CONST{#2}}
\def\SEG#1#2#3{\CONST{SEG}\,\CONST{#1}\,\CONST{#2}\,#3}
\def\T{\CONST{T}}

%\def\dotoken#1#2{\mbox{#1#2}\endgroup}
%\def\idxname#1{\begingroup\makeulother\dotoken#1}
%\def\dotokenidx#1#2{\mbox{#1#2}\index{#2@\string\idxname{#1}{#2}}\endgroup}
%\def\mlname{\begingroup\makeulother\dotokenidx{\tt}}
%\def\CONST{\begingroup\makeulother\dotokenidx{\constfont}}
%\def\KEYWD{\begingroup\makeulother\dotoken{\keyfont}}
%
%\def\idxmlname{\begingroup\makeulother\dotoken{\tt}}
%\def\idxconst{\begingroup\makeulother\dotoken{\constfont}}
%\def\ul#1{\relax\ifmmode\underline#1\else$\underline{#1}$\fi}

% define environment for HOL definitions and theorems
\def\makeulother{\catcode`\_=12\relax}
\def\makeulsub{\catcode`\_=8\relax}
%\begingroup
% \makeulother
% \gdef\_{\ul}
%\endgroup
\def\doholdef#1{\par\vspace*{5pt}\index{#1@\string\idxname{\tt}{#1}|ul}%
 \flushleft{\bf HOL Definition }({\tt#1})\label{def-#1}\endgroup}
\def\holdef{\begingroup\makeulother\doholdef}
\let\endholdef=\endflushleft

\def\doholthm#1{\par\index{#1@\string\idxname{\tt}{#1}|ul}%
 \flushleft{\bf HOL Theorem }({\tt#1})\label{thm-#1}\endgroup}
\def\holthm{\begingroup\makeulother\doholthm}
\let\endholthm=\endflushleft

\font\sfc=cmssc12 \def\constfont{\sfc}
\def\ul#1{$\underline{#1}$}