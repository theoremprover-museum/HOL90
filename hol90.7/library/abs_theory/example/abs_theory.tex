\documentstyle[a4]{article}
% pargap.sty
% implement a non-indented with gaps between paragraphs style
%
% david shepherd
% 30may89
%
% nb various odds and ends in some styles may take some time before
% they get spotted

\parindent=0pt
\parskip=0.5\baselineskip plus 1pt
\listparindent=0pt

% need to change parindent and skip in footnote macro
% [DES] 15jun89
\long\def\@makefntext#1{\parindent 0pt\parskip 0.5\baselineskip
 \noindent 
 \hbox to 1.8em{\hss$^{\@thefnmark}$}#1}

% no indents in quotations -- esp abtract !!
% [DES] 7aug89
\def\quotation{\list{}{\listparindent 0pt
 \itemindent\listparindent
 \rightmargin\leftmargin\parsep 0.5\baselineskip plus 1pt}\item[]}

% no indents + gap in parboxen
\def\@arrayparboxrestore{\let\par\@@par
    \let\-\@dischyph
    \let\'\@acci \let\`\@accii \let\=\@acciii
    \parindent\z@
    \parskip=0.5\baselineskip plus 1pt
    \everypar{}\linewidth\hsize 
    \@totalleftmargin\z@ \leftskip\z@ \rightskip\z@ \@rightskip\z@ 
    \parfillskip\@flushglue \lineskip\normallineskip 
    \baselineskip\normalbaselineskip\sloppy}

\title{Abstract Theories for HOL90}
\author{David Shepherd\thanks{All the credit for this really belongs to
Phil Windley, all I have done is to move his Hol88 package into Hol90
with a bit of re-engineering and what I consider to be tidying up!}}
\date{February 24, 1993}

\begin{document}
\maketitle

\begin{abstract}
This paper serves to document my port of the abstract theory package
into Hol90. Rather than give proper ``Manual'' style entries, I am just
annotating a couple of SML files that illustrate all the bits of the
package in action. We will do this by using the traditional example of 
monoids and groups! For a motivation for and description of abstract
theories please refer to the paper by Phil Windley in the Hol88 abstract
theory package or in the proceedings of HOL-92.
\end{abstract}

\section{Monoids}

To start off with we need to load the package in, open the
structure containing the definitions and declare a new theory.
N.b. unless relevent I will omit the responses from the system.

\begin{small}
\begin{verbatim}
   - load_library_in_place abs_theory_lib;

   - open Abs_theory Goalstack;

   - new_theory "monoid";
\end{verbatim}
\end{small}

The first thing we do is to declare a new ``abstract entity''. This
declaration combines the declaration of an abstract representation and
obligations which in the Hol88 version were done separately. The
declaration has two parameters, the name of the new entity (here
``monoid'') and a predicate containing the obligations. The type
operations are derived from all the free variables in this predicate.

\begin{small}
\begin{verbatim}
- val monoid_th = new_abstract_entity
    {Name="monoid",
     Predicate=(--`(! x. op x (e:'a) = x) /\
                   (! x. op e x = x) /\
                   (! x y z. (op x (op y z)) = (op (op x y) z))`--)};

val monoid_th = |- !f. ?!fn. !x f'. fn (monoid x f') = f x f' : thm
\end{verbatim}
\end{small}  

Next we define a couple of auxiliary functions, firstly a derived rule
that's missing from Hol90 and secondly a function to save a theorem and
bind it to an SML variable at the same time.

\begin{small}
\begin{verbatim}
- val SYM_RULE = CONV_RULE(ONCE_DEPTH_CONV SYM_CONV);

- fun define_theorem name th =
    (save_thm (name,th);
     add_to_sml[(name,th)]);
\end{verbatim}
\end{small}


When proving theorems the approach I have taken is to set the goal by
universally quantifying all the abstract terms without including any theory
obligations. The abstract theory versions of various functions then
put in these obligations. This simplifies the matter of specifying
goals.

Here we set a goal to prove that the monoid operation defines the identity.

\begin{small}
\begin{verbatim}
- abs_set_goal([],(--`! (m: 'a monoid) f.
                      (!a. (op m a f = a) /\ (op m f a = a))
                      ==> (f = (e m))`--));

(--`!m.
      monoid_oblig m ==>
      (!f. (!a. (op m a f = a) /\ (op m f a = a)) ==> (f = e m))`--)
------------------------------------

val it = () : unit
\end{verbatim}
\end{small}

Note how the monoid obligation was added to the goal. The goal is now
in what I define to be the normal form for abstract terms --- all
abstract entities are universally quantified at the outside of an
implication whose antecedent is a conjunction of the obligations for
all these abstract entities.

To continue with the proof we first strip off all the outer universal
quantifications and obligations using the tactic
\verb!STRIP_THOBS_TAC!.

\begin{small}
\begin{verbatim}
- expand(STRIP_THOBS_TAC);

OK..
1 subgoal:
(--`!f. (!a. (op m a f = a) /\ (op m f a = a)) ==> (f = e m)`--)
=============================
  (--`monoid_oblig m`--)
  (--`!x. op m x (e m) = x`--)
  (--`!x. op m (e m) x = x`--)
  (--`!x y z. op m x (op m y z) = op m (op m x y) z`--)
\end{verbatim}
\end{small}

Note that this places both the obligation {\em and} its expansion
on the assumption list. The reason for this will become apparent later.
There is also a related tactic continuation \verb!STRIP_THOBS_THEN! which takes
two theorem tactics as arguments to define what to do with firstly the
unexpanded obligations and then the expanded obligations.
After this the proof is the same as in the Hol88 version and it
can all be gathered together into

\begin{small}
\begin{verbatim}
- define_theorem "IDENTITY_UNIQUE"
  (abs_prove((--`! (m:'a monoid) f .
                   (!a .(op m a f = a) /\ (op m f a = a)) ==> (f = (e m))`--),
             STRIP_THOBS_TAC
             THEN REPEAT GEN_TAC
             THEN STRIP_GOAL_THEN
                    (SUBST1_TAC o SYM_RULE
                     o CONJUNCT1 o SPEC (--`e (m:'a monoid)`--))
             THEN ASM_REWRITE_TAC []
             ));


val IDENTITY_UNIQUE =
  |- !m.
       monoid_oblig m ==>
       (!f. (!a. (op m a f = a) /\ (op m f a = a)) ==> (f = e m)) : thm
\end{verbatim}
\end{small}

\verb!abs_prove! is to \verb!prove! what \verb!abs_set_goal! is to
\verb!set_goal! in that it adds the abstract obligations to the term it
proves. Similarily there is an \verb!abs_store_thm! which proves and
saves a theorem about abstract terms.

Another simple proof which follows the same pattern is:
    
\begin{small}
\begin{verbatim}
- define_theorem "OP_DETERMINES_IDENTITY"
  (abs_prove
   ( --`! m1 (m2:'a monoid) . (op m1 = (op m2)) ==> (e m1 = (e m2))`--,
    STRIP_THOBS_TAC
    THEN STRIP_TAC
    THEN (let val t1 = ASSUME (--`!x:'a. op m1 (e m1) x = x`--)
          in SUBST_TAC (map SYM_RULE [SPEC (--`e m2:'a`--) t1])
          end)
    THEN (
          let val t2 = ASSUME (--`!x:'a. op m2 x (e m2) = x`--)
          in SUBST_TAC (map SYM_RULE [SPEC(--`e m1:'a`--) t2])
          end)
    THEN ASM_REWRITE_TAC [])
   );
  
val OP_DETERMINES_IDENTITY =
  |- !m1 m2.
       monoid_oblig m1 /\ monoid_oblig m2 ==> (op m1 = op m2) ==> (e m1 = e m2)
  : thm
\end{verbatim}
\end{small}

Finally we define a squaring operation on monoids.

\begin{small}
\begin{verbatim}
- new_definition("monoid_sq",
                 (--`monoid_sq (m:'a monoid) a = (op m a a)`--));

val it = |- !m a. monoid_sq m a = op m a a : thm
\end{verbatim}
\end{small}

And save the theory:

\begin{small}
\begin{verbatim}
- export_theory();
\end{verbatim}
\end{small}

That concludes the simple theory of monoids. 

\section{Groups}

Next we assume that we wish to start a new Hol90 session and develop a
theory of groups. Again we load the package, define the auxiliary
functions and open the theory.

\begin{small}
\begin{verbatim}
- load_library_in_place abs_theory_lib;
- open Abs_theory Goalstack;
    
- fun define_theorem name th =
    (save_thm (name,th);
     add_to_sml[(name,th)]);

- val SYM_RULE = CONV_RULE(ONCE_DEPTH_CONV SYM_CONV);

- new_theory "groups";
\end{verbatim}
\end{small}

We first declare the monoid theory to be a parent

\begin{small}
\begin{verbatim}
- new_parent "monoid";
\end{verbatim}
\end{small}

Next we add all the abstract representation and obligation information
from that theory into the current state -- without doing this
various subsequent steps will fail when the system cannot find the
relevant information about the obligations of the monoid subtype.

\begin{small}
\begin{verbatim}
- add_obligs "monoid";
\end{verbatim}
\end{small}

While \verb!add_obligs! could be built into a \verb!new_abs_parent!
function, it seems better to force that obligations are explicitly
loaded. The reason for this is that any parent declaration may bring
in abstract theories from higher up the theory hierachy. Hence for
consistency, either every parent declaration needs to add obligations
from all its ancestors, which is time consuming, or we force
obligations to be explicitly loaded by the user.

The final set up stage is to load the theorems we will use out of the
monoid theory. The function \verb!abs_add_theory! is a version of
\verb!add_theory! which adds binds everything in the theory to SML
variables apart from the definitions used to define the abstract
representation (i.e. the obligation theory, the term represenation
definitions and the definition theorems for the representing type. N.b.
these are in fact de-selected by suffix of the definition name so
actually any definition with suffix \verb!_DEF!, \verb!_oblig! or
\verb!_abs_rep! will be skipped by this!)

\begin{small}
\begin{verbatim}
- abs_add_theory_to_sml "monoid";
\end{verbatim}
\end{small}

Now we can define the abstract entity for groups.

\begin{small}
\begin{verbatim}
- new_abstract_entity 
  {Name="group",
   Predicate=(--`(!x .fn x (id:'a) = x) /\
                 (!x .fn id x = x) /\
                 (!x .fn x (inv x) = id ) /\
                 (!x. fn (inv x) x = id) /\
                 (!x y z. (fn x (fn y z)) = (fn (fn x y) z))`--)
   };
\end{verbatim}
\end{small}

The first thing that we prove is that a group is a monoid. We set up the
goal and strip off the group obligations. We set up the goal using
\verb!ag! which is an abstract analogue to \verb!g!.

\begin{small}
\begin{verbatim}
- ag`! g:'a group . monoid_oblig(monoid (id g) (fn g))`;

(--`!g. group_oblig g ==> monoid_oblig (monoid (id g) (fn g))`--)
=============================

- expand(STRIP_THOBS_TAC)
OK..
1 subgoal:
(--`monoid_oblig (monoid (id g) (fn g))`--)
=============================
  (--`group_oblig g`--)
  (--`!x. fn g x (id g) = x`--)
  (--`!x. fn g (id g) x = x`--)
  (--`!x. fn g x (inv g x) = id g`--)
  (--`!x. fn g (inv g x) x = id g`--)
  (--`!x y z. fn g x (fn g y z) = fn g (fn g x y) z`--)
\end{verbatim}
\end{small}

The is a conversion in the package that will expand an abstract obligation
with rule and tactic versions designed for cases like this. We apply this
here.

\begin{small}
\begin{verbatim}
- expand(EXPAND_THOBS_TAC);
OK..
1 subgoal:
(--`(!x. fn g x (id g) = x) /\
    (!x. fn g (id g) x = x) /\
    (!x y z. fn g x (fn g y z) = fn g (fn g x y) z)`--)
=============================
  (--`group_oblig g`--)
  (--`!x. fn g x (id g) = x`--)
  (--`!x. fn g (id g) x = x`--)
  (--`!x. fn g x (inv g x) = id g`--)
  (--`!x. fn g (inv g x) x = id g`--)
  (--`!x y z. fn g x (fn g y z) = fn g (fn g x y) z`--)
\end{verbatim}
\end{small}

This rewrites first with the relevant obligation theorem and then with the
representation definitions for that abstract entity. The proof is now
simply rewriting with the assumptions.

\begin{small}
\begin{verbatim}
- val GROUP_EXTENDS_MONOID = 
    abs_prove((--`! g:'a group . monoid_oblig(monoid (id g) (fn g))`--),
              STRIP_THOBS_TAC
              THEN EXPAND_THOBS_TAC
              THEN ASM_REWRITE_TAC []
              );

val GROUP_EXTENDS_MONOID =
  |- !g. group_oblig g ==> monoid_oblig (monoid (id g) (fn g)) : thm
\end{verbatim}
\end{small}

Now that we know that a group is a monoid, we can lift the monoid
theorems into group theory. \verb!instantiate_abs_thm! takes as its
arguments an instatiation list and an theorem. The instantiation list is a
list of instantiations consisting of the abstract variable in the
theorem to be instaniated, a term which contains the new representation
and a validation that the new representation satisfies the relevant
obligation. In these examples the representation is a monoid using the
group identity and operation.

\begin{small}
\begin{verbatim}
- define_theorem "IDENTITY_UNIQUE"
  (instantiate_abs_thm[{abs_term=(--`m:'a monoid`--),
                        rep_term=(--`monoid (id(g:'a group))(fn g)`--),
                        validation=GROUP_EXTENDS_MONOID}]
                      IDENTITY_UNIQUE);

val IDENTITY_UNIQUE =
  |- !g.
       group_oblig g ==>
       (!f. (!a. (fn g a f = a) /\ (fn g f a = a)) ==> (f = id g)) : thm

- define_theorem "OP_DETERMINES_IDENTITY"
  (instantiate_abs_thm[{abs_term=(--`m1:'a monoid`--),
                        rep_term=(--`monoid (id(g1:'a group))(fn g1)`--),
                        validation=GROUP_EXTENDS_MONOID},
                       {abs_term=(--`m2:'a monoid`--),
                        rep_term=(--`monoid (id(g2:'a group))(fn g2)`--),
                        validation=GROUP_EXTENDS_MONOID}]
                    OP_DETERMINES_IDENTITY);
  
val OP_DETERMINES_IDENTITY =
  |- !g2 g1.
       group_oblig g1 /\ group_oblig g2 ==> (fn g1 = fn g2) ==> (id g1 = id g2)
  : thm
\end{verbatim}
\end{small}

Notice how the monoid obligations have turned into group obligations.
Lifting definitions into the new theory is slightly more complex as we take
2 steps. Firstly we define the new definition in terms of the old one

\begin{small}
\begin{verbatim}
- val group_sq =
    new_definition("group_sq",
                   (--`group_sq (g:'a group) a 
                       = (monoid_sq (monoid(id g)(fn g)) a)`--));

val group_sq = |- !g a. group_sq g a = monoid_sq (monoid (id g) (fn g)) a : thm
\end{verbatim}
\end{small}

Then we instantiate the definition using an instantiation list (without the
validation this time as definitions are not guarded by obligations), the old
definition and the new definiiton

\begin{small}
\begin{verbatim}
- val group_sq_thm 
    = instantiate_abstract_definition 
        [{abs_term=(--`m:'a monoid`--),
          rep_term=(--`monoid(id (g:'a group))(fn g)`--)}]
        monoid_sq group_sq;

val group_sq_thm = |- !a g. group_sq g a = fn g a a : thm
\end{verbatim}
\end{small}

Of course, in this trivial example it would have been simpler to make this
definition directly!.

We can prove theorems of group theory just as we did in monoid theory.

\begin{small}
\begin{verbatim}
- define_theorem "LEFT_CANCELLATION"
   (abs_prove
     (--`! (g:'a group) x y a. ((fn g) a x = ((fn g) a y)) ==> (x = y)`--,
      STRIP_THOBS_TAC
      THEN REPEAT STRIP_TAC
      THEN ACCEPT_TAC 
      let val t1 = ASSUME(--`!x y z. fn (g:'a group) x(fn g y z)
			     = fn g(fn g x y)z`--)
	  and t2 = ASSUME (--`!x. fn (g:'a group)(inv g x)x = id g`--)
	  and t3 = ASSUME (--`!x. fn (g:'a group)(id g) x  = x`--)
	  and t4 = ASSUME (--`fn (g:'a group) a x = fn g a y`--)
      in
      SYM_RULE (REWRITE_RULE [t1,t2,t3] 
		(REWRITE_RULE [t2,t3,t4] 
		 (ISPECL[(--`(inv g (a:'a))`--),(--`a:'a`--),(--`x:'a`--)]t1)))
      end));

val LEFT_CANCELLATION =
  |- !g. group_oblig g ==> (!x y a. (fn g a x = fn g a y) ==> (x = y))
  : thm

- define_theorem "INVERSE_INVERSE_LEMMA"
  (abs_prove
    (--`!(g:'a group). !a. (((inv g) ((inv g) a)) = a)`--,
     STRIP_THOBS_TAC
     THEN GEN_TAC
     THEN ACCEPT_TAC 
     let val t1 = ASSUME(--`!x. fn (g:'a group) x(inv g x) = id g`--)
	 and t2 = ASSUME(--`!x. fn (g:'a group) (inv g x)x = id g`--)
	 and LC_LEMMA = (ISPECL [--`inv (g:'a group) (inv g a)`--,
				 --`a:'a`--,--`inv g (a:'a)`--]
			 o UNDISCH o ISPEC(--`g:'a group`--))
			LEFT_CANCELLATION 
     in
     MATCH_MP LC_LEMMA (TRANS (ISPEC(--`(inv g)(a:'a)`--)t1) 
			(SYM_RULE (ISPEC (--`a:'a`--)t2)))
     end));

val INVERSE_INVERSE_LEMMA = 
  |- !g. group_oblig g ==> (!a. inv g (inv g a) = a)
  : thm
\end{verbatim}
\end{small}

This last theory shows us that a double inverse can be removed. This is
something we clearly would like to use as a rewrite rule. If we
instantiate the theorem with a concrete representation (for example
integers and plus) then we clearly have a theorem (effectively 
\verb|!a. - (- a) = a|). However while in the abstract theory we cannot 
do this as we have the group obligation to deal with. This can be 
handled by suitable resolutions but it would be nice to be able to treat
it as a simple rewrite rule. To do this there is in addition a
structure of rewriting conversions, rules and tactics for theorems with
preconditions. We load this into the system now.

\begin{small}
\begin{verbatim}
- open Imp_rewrite;
\end{verbatim}
\end{small}

For any existing rewrite rule there is an new one with \verb!IMP_! prefixed
to \verb!REWRITE! in its name. These rules strip off an outer levels of
universal quantification and implication by generalising quantifications
and undischarging antecentedents in the assumption lists. The resulting
theorems are used as rewrites with the condition that they are not
allowed to introduce any new assumptions into the theorem or goal --- the
conversions have an additional parameter listing the assumptions that
may be created. To show this in action we set up a simple goal

\begin{small}
\begin{verbatim}
- ag `! (g:num group) a. inv g(inv g(inv g a)) = inv g a`;

(--`!g. group_oblig g ==> (!a. inv g (inv g (inv g a)) = inv g a)`--)
=============================
\end{verbatim}
\end{small}

We strip off the obligations as before

\begin{small}
\begin{verbatim}
- expand(STRIP_THOBS_TAC);

OK..
1 subgoal:
(--`!a. inv g (inv g (inv g a)) = inv g a`--)
=============================
  (--`group_oblig g`--)
  (--`!x. fn g x (id g) = x`--)
  (--`!x. fn g (id g) x = x`--)
  (--`!x. fn g x (inv g x) = id g`--)
  (--`!x. fn g (inv g x) x = id g`--)
  (--`!x y z. fn g x (fn g y z) = fn g (fn g x y) z`--)
\end{verbatim}
\end{small}

We can now rewrite this with \verb!INVERSE_INVERSE_LEMMA!. Note that
the relevant obligation is in the assumptions (so the rewrite will work)
--- this was the reason for putting both the unexpanded and expanded
obligations on the assumption list.

\begin{small}
\begin{verbatim}
- expand(PURE_ONCE_IMP_REWRITE_TAC[INVERSE_INVERSE_LEMMA]);
OK..
1 subgoal:
(--`!a. inv g a = inv g a`--)
=============================
  (--`group_oblig g`--)
  (--`!x. fn g x (id g) = x`--)
  (--`!x. fn g (id g) x = x`--)
  (--`!x. fn g x (inv g x) = id g`--)
  (--`!x. fn g (inv g x) x = id g`--)
  (--`!x y z. fn g x (fn g y z) = fn g (fn g x y) z`--)
\end{verbatim}
\end{small}

And the rest of the proof is simple:

\begin{small}
\begin{verbatim}       
- abs_prove((--`! (g:num group) a. inv g(inv g(inv g a)) = inv g a`--),
          STRIP_THOBS_TAC
          THEN IMP_REWRITE_TAC[INVERSE_INVERSE_LEMMA]);

val it = |- !g. group_oblig g ==> (!a. inv g (inv g (inv g a)) = inv g a) : thm
\end{verbatim}
\end{small}

This also demonstrates the advantage of having a normal form for theorems
with abstract types which allows support functions to be built to handle them.

Finally we save the theory

\begin{small}
\begin{verbatim}
val _ = export_theory();
\end{verbatim}
\end{small}

\section{Booleans and exclusive-or}

We now provide a concrete example of a group. We start the theory

\begin{small}
\begin{verbatim}
- load_library_in_place abs_theory_lib;
- open Abs_theory Goalstack;
    
- new_theory "exorgroup";

- new_parent"groups";

- add_obligs "groups";
\end{verbatim}
\end{small}

and then we define exclusive-or by

\begin{small}
\begin{verbatim}
- val exclusive_or =
  new_infix_definition("exclusive_or",
                       (--`! (x:bool) y. $>< x y = ~(x=y)`--),20);

val exclusive_or = |- !x y. (x >< y) = ~(x = y) : thm
\end{verbatim}
\end{small}

N.b. I'm not sure if the fixity value is sensible!

First we need to show that we have a group with exclusive-or as the
operation, false as the identity and the identity function as the inverse!.

\begin{small}
\begin{verbatim}
- g`group_oblig(group I F ($><))`;

- expand(EXPAND_THOBS_TAC);
OK..
1 subgoal:
(--`(!x. (x >< F) = x) /\
    (!x. (F >< x) = x) /\
    (!x. (x >< I x) = F) /\
    (!x. (I x >< x) = F) /\
    (!x y z. (x >< y >< z) = ((x >< y) >< z))`--)
=============================
\end{verbatim}
\end{small}

Now we can rewrite with existing definitions and prove the goal:

\begin{small}
\begin{verbatim}
- val exorgroup =
  prove((--`group_oblig(group I F ($><))`--),
        EXPAND_THOBS_TAC
        THEN REWRITE_TAC[exclusive_or,theorem"combin""I_THM"]
        THEN (REPEAT GEN_TAC
              THEN MAP_EVERY BOOL_CASES_TAC [--`x:bool`--,
                                             --`y:bool`--,
                                             --`z:bool`--]
              THEN REWRITE_TAC[]));
  
val exorgroup = |- group_oblig (group I F $><) : thm
\end{verbatim}
\end{small}

So now we can lift the group theorems.
  
\begin{small}
\begin{verbatim}
- val EXOR_IDENTITY_UNIQUE =
    instantiate_abs_thm [{abs_term=(--`g:'a group`--),
                          rep_term=(--`group I F ($><)`--),
                          validation=exorgroup}]
                        IDENTITY_UNIQUE;

val EXOR_IDENTITY_UNIQUE =
  |- !f. (!a. ((a >< f) = a) /\ ((f >< a) = a)) ==> (f = F) : thm

- val EXOR_GROUP_ID =       
    instantiate_abs_thm [{abs_term=(--`g2:'a group`--),
                          rep_term=(--`group I F ($><)`--),
                          validation=exorgroup}]
                        OP_DETERMINES_IDENTITY;

val EXOR_GROUP_ID = |- !g1. group_oblig g1 ==> (fn g1 = $><) ==> (id g1 = F)
  : thm

- val EXOR_LEFT_CANCELLATION =
    instantiate_abs_thm [{abs_term=(--`g:'a group`--),
                          rep_term=(--`group I F ($><)`--),
                          validation=exorgroup}]
                        LEFT_CANCELLATION;

val EXOR_LEFT_CANCELLATION = |- !x y a. ((a >< x) = (a >< y)) ==> (x = y) : thm
    
- val EXOR_INVERSE_INVERSE_LEMMA =
    instantiate_abs_thm [{abs_term=(--`g:'a group`--),
                          rep_term=(--`group I F ($><)`--),
                          validation=exorgroup}]
                        INVERSE_INVERSE_LEMMA;

val EXOR_INVERSE_INVERSE_LEMMA = |- !a. I (I a) = a : thm
\end{verbatim}
\end{small}

Note that in the second case we do not instantiate all the abstract
variables. Also note that we have 4 theorems here of which the second
two were proved in terms of general groups while the first two originate
from proofs about monoids. Simply by showing that a group can be viewed
as a monoid and that exclusive-or, not and false together form a group
has enabled us to derive theorems in the boolean domain from theorems
about generic monoids. In this case the saving may not seem readily
apparent, however, if a fuller group theory was developed then a whole
body of theorems about groups could be instantiated both inside ring
theory as well as in concrete cases such as arithmetic over integers,
rationals, reals etc. The proofs are only done once in group theory with
each instantiation only requiring proof of an instance of a group in
terms of other abstract or concrete entities.

\end{document}
