\chapter{ML Functions in the taut Library}
This chapter provides documentation on all the \ML\ functions that are made
available in \HOL\ when the \ml{res\_quan} library is loaded.  This documentation is
also available online via the \ml{help} facility.



\DOC{PTAUT\_CONV}

\TYPE {\small\verb%PTAUT_CONV : conv%}\egroup

\SYNOPSIS
Tautology checker. Proves closed propositional formulae true or false.

\DESCRIBE
Given a term of the form {\small\verb%"!x1 ... xn. t"%} where {\small\verb%t%} contains only Boolean
constants, Boolean-valued variables, Boolean equalities, implications,
conjunctions, disjunctions, negations and Boolean-valued conditionals, and
all the variables in {\small\verb%t%} appear in {\small\verb%x1 ... xn%}, the conversion {\small\verb%PTAUT_CONV%}
proves the term to be either true or false, that is, one of the following
theorems is returned:
{\par\samepage\setseps\small
\begin{verbatim}
   |- (!x1 ... xn. t) = T
   |- (!x1 ... xn. t) = F
\end{verbatim}
}
\noindent This conversion also accepts propositional terms that are not fully
universally quantified. However, for such a term, the conversion will only
succeed if the term is valid.

\FAILURE
Fails if the term is not of the form {\small\verb%"!x1 ... xn. f[x1,...,xn]"%} where
{\small\verb%f[x1,...,xn]%} is a propositional formula (except that the variables do not
have to be universally quantified if the term is valid).

\EXAMPLE
{\par\samepage\setseps\small
\begin{verbatim}
#PTAUT_CONV "!x y z w. (((x \/ ~y) ==> z) /\ (z ==> ~w) /\ w) ==> y";;
|- (!x y z w. (x \/ ~y ==> z) /\ (z ==> ~w) /\ w ==> y) = T

#PTAUT_CONV "(((x \/ ~y) ==> z) /\ (z ==> ~w) /\ w) ==> y";;
|- (x \/ ~y ==> z) /\ (z ==> ~w) /\ w ==> y = T

#PTAUT_CONV "!x. x = T";;
|- (!x. x = T) = F

#PTAUT_CONV "x = T";;
evaluation failed     PTAUT_CONV
\end{verbatim}
}
\SEEALSO
PTAUT_PROVE, PTAUT_TAC, TAUT_CONV.

\ENDDOC
\DOC{PTAUT\_PROVE}

\TYPE {\small\verb%PTAUT_PROVE : conv%}\egroup

\SYNOPSIS
Tautology checker. Proves propositional formulae.

\DESCRIBE
Given a term that contains only Boolean constants, Boolean-valued variables,
Boolean equalities, implications, conjunctions, disjunctions, negations and
Boolean-valued conditionals, {\small\verb%PTAUT_PROVE%} returns the term as a theorem if it
is valid. The variables in the term may be universally quantified.

\FAILURE
Fails if the term is not a valid propositional formula.

\EXAMPLE
{\par\samepage\setseps\small
\begin{verbatim}
#PTAUT_PROVE "!x y z w. (((x \/ ~y) ==> z) /\ (z ==> ~w) /\ w) ==> y";;
|- !x y z w. (x \/ ~y ==> z) /\ (z ==> ~w) /\ w ==> y

#PTAUT_PROVE "(((x \/ ~y) ==> z) /\ (z ==> ~w) /\ w) ==> y";;
|- (x \/ ~y ==> z) /\ (z ==> ~w) /\ w ==> y

#PTAUT_PROVE "!x. x = T";;
evaluation failed     PTAUT_PROVE

#PTAUT_PROVE "x = T";;
evaluation failed     PTAUT_PROVE
\end{verbatim}
}
\SEEALSO
PTAUT_CONV, PTAUT_TAC, TAUT_PROVE.

\ENDDOC
\DOC{PTAUT\_TAC}

\TYPE {\small\verb%PTAUT_TAC : tactic%}\egroup

\SYNOPSIS
Tautology checker. Proves propositional goals.

\DESCRIBE
Given a goal with a conclusion that contains only Boolean constants,
Boolean-valued variables, Boolean equalities, implications, conjunctions,
disjunctions, negations and Boolean-valued conditionals, this tactic will
prove the goal if it is valid. If all the variables in the conclusion are
universally quantified, this tactic will also reduce an invalid goal to false.

\FAILURE
Fails if the conclusion of the goal is not of the form
{\small\verb%"!x1 ... xn. f[x1,...,xn]"%} where {\small\verb%f[x1,...,xn]%} is a propositional formula
(except that the variables do not have to be universally quantified if the
goal is valid).

\SEEALSO
PTAUT_CONV, PTAUT_PROVE, TAUT_TAC.

\ENDDOC
\DOC{TAUT\_CONV}

\TYPE {\small\verb%TAUT_CONV : conv%}\egroup

\SYNOPSIS
Tautology checker. Proves instances of propositional formulae.

\DESCRIBE
Given an instance of a valid propositional formula, {\small\verb%TAUT_CONV%} proves the
instance of the formula to be true. A propositional formula is a term
containing only Boolean constants, Boolean-valued variables, Boolean
equalities, implications, conjunctions, disjunctions, negations and
Boolean-valued conditionals. An instance of a formula is the formula with one
or more of the variables replaced by terms of the same type. The conversion
accepts terms with or without universal quantifiers for the variables.

\FAILURE
Fails if the term is not an instance of a propositional formula or if the
instance is not a valid formula.

\EXAMPLE
{\par\samepage\setseps\small
\begin{verbatim}
#TAUT_CONV
# "!x n y. ((((n = 1) \/ ~x) ==> y) /\ (y ==> ~(n < 0)) /\ (n < 0)) ==> x";;
|- (!x n y. ((n = 1) \/ ~x ==> y) /\ (y ==> ~n < 0) /\ n < 0 ==> x) = T

#TAUT_CONV "((((n = 1) \/ ~x) ==> y) /\ (y ==> ~(n < 0)) /\ (n < 0)) ==> x";;
|- ((n = 1) \/ ~x ==> y) /\ (y ==> ~n < 0) /\ n < 0 ==> x = T

#TAUT_CONV "!n. (n < 0) \/ (n = 0)";;
evaluation failed     TAUT_CONV
\end{verbatim}
}
\SEEALSO
TAUT_PROVE, TAUT_TAC, PTAUT_CONV.

\ENDDOC
\DOC{TAUT\_PROVE}

\TYPE {\small\verb%TAUT_PROVE : conv%}\egroup

\SYNOPSIS
Tautology checker. Proves propositional formulae (and instances of them).

\DESCRIBE
Given an instance of a valid propositional formula, {\small\verb%TAUT_PROVE%} returns the
instance of the formula as a theorem. A propositional formula is a term
containing only Boolean constants, Boolean-valued variables, Boolean
equalities, implications, conjunctions, disjunctions, negations and
Boolean-valued conditionals. An instance of a formula is the formula with one
or more of the variables replaced by terms of the same type. The conversion
accepts terms with or without universal quantifiers for the variables.

\FAILURE
Fails if the term is not an instance of a propositional formula or if the
instance is not a valid formula.

\EXAMPLE
{\par\samepage\setseps\small
\begin{verbatim}
#TAUT_PROVE
# "!x n y. ((((n = 1) \/ ~x) ==> y) /\ (y ==> ~(n < 0)) /\ (n < 0)) ==> x";;
|- !x n y. ((n = 1) \/ ~x ==> y) /\ (y ==> ~n < 0) /\ n < 0 ==> x

#TAUT_PROVE "((((n = 1) \/ ~x) ==> y) /\ (y ==> ~(n < 0)) /\ (n < 0)) ==> x";;
|- ((n = 1) \/ ~x ==> y) /\ (y ==> ~n < 0) /\ n < 0 ==> x

#TAUT_PROVE "!n. (n < 0) \/ (n = 0)";;
evaluation failed     TAUT_PROVE
\end{verbatim}
}
\SEEALSO
TAUT_CONV, TAUT_TAC, PTAUT_PROVE.

\ENDDOC
\DOC{TAUT\_TAC}

\TYPE {\small\verb%TAUT_TAC : tactic%}\egroup

\SYNOPSIS
Tautology checker. Proves propositional goals (and instances of them).

\DESCRIBE
Given a goal that is an instance of a propositional formula, this tactic will
prove the goal provided it is valid. A propositional formula is a term
containing only Boolean constants, Boolean-valued variables, Boolean
equalities, implications, conjunctions, disjunctions, negations and
Boolean-valued conditionals. An instance of a formula is the formula with one
or more of the variables replaced by terms of the same type. The tactic
accepts goals with or without universal quantifiers for the variables.

\FAILURE
Fails if the conclusion of the goal is not an instance of a propositional
formula or if the instance is not a valid formula.

\SEEALSO
TAUT_CONV, TAUT_PROVE, PTAUT_TAC.

\ENDDOC
