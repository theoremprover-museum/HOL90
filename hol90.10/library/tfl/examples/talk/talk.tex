\documentstyle[fancybox,sem-a4,itmath]{seminar}
\input{amssym.def}
\input{amssym.tex}

\landscapeonly 
%\portraitonly

% Personal hints concerning printing:
% 1. enable \landscapeonly and print by 
%    dvips -t landscape slides
% 2. enable \portraitonly and use portrait-option to print the rest by
%    dvips slides 

%\slidesmag{2}

\newcommand{\heading}[1]{%
  \begin{center}
    \large\bf
    \shadowbox{\makebox{#1}}%
  \end{center}
  \vspace{1ex minus 1ex}}

\newcommand{\Infer}[2]%
{\frac{\vbox{\hbox{$\displaystyle #1$}\hbox{\vspace{.1ex}}}}%
{\vbox{\hbox{\vspace{.3ex}}\hbox{$\displaystyle #2$}}}}


\begin{document}


\begin{slide}
\heading{Defining Recursive Functions}
\heading{in Higher Order Logic}
\bigskip
\bigskip
\begin{center}
  Konrad Slind  \\
TU M{\"u}nchen
\end{center}
\end{slide} 

\begin{slide}
\heading{Motivation}
\begin{itemize}
\item Adding an axiom for a non-terminating function to a logic of total
      functions results in inconsistency: 

\[ f(x) = f(x) + 1 \]

reduces to 

\[ 0 = 1 \]

\item Termination is undecidable in general.
\item Primitive recursion imposes syntactic restrictions.
\end{itemize}
\end{slide}

\begin{slide}
\heading{Plan}

\begin{enumerate}
\item Define functions
\begin{itemize}
\item Logical Basis (wellfoundedness, recursion, induction)
\item Definition Process
   \begin{itemize}
   \item Adding pattern matching
   \end{itemize}
\item Examples
\end{itemize}
\item Custom induction theorems
\item System Architecture
\item Future Work
\end{enumerate}
\end{slide} 


\begin{slide}
\heading{Logical Basis}
\medskip
\begin{itemize}
\item Wellfoundedness

\begin{eqnarray*}
wellfounded(R) & \equiv & \neg\exists f. \forall n. R (f(n+1)) (f(n)) \\
          WF(R) & \equiv & \forall P.\ P \;w \supset \exists m.\;
   P\;m \land \forall b.\; R \;b \; m \supset \neg P\;b
\end{eqnarray*}


%\begin{verbatim}
%well_founded R = ~?f. !n. R (f (n + 1)) (f n)
%WF R = !B. (?w. B w) ==> 
%           ?min. B min /\ !b. R b min ==> ~(B b)
%\end{verbatim}

%\item Induction
%\begin{verbatim}
%WF R ==> (!x. (!y. R y x ==> P y) ==> P x) ==> !x. P x
%\end{verbatim}

\item Induction
\begin{eqnarray*}
 WF(R)  \supset (\forall x.\; (\forall y.\;R\; y\; x \supset  P\; y) \supset P
\;x) \supset \forall x.\; P \;x.
\end{eqnarray*}


\item Restriction and Recursion
\begin{eqnarray*}
 (f\!\mid\! R,y) \equiv \lambda x.\; \mbox{if}\; R\;x \;y
\;\mbox{then}\; f\; x \; \mbox{else}\; \varepsilon z.True. \\
(f = WFREC\ R\ M) \supset WF(R) \supset \forall x.\ f(x) = M\ (f\!\mid\!
R,x)\ x.
\end{eqnarray*}

\end{itemize}

%\begin{verbatim}
%(f % R,y) = \x. if (R x y) then (f x) else @v. T
%
%WF R ==> !M. ?!f. !x. f x = M x (f % R,x)
%(f = WFREC R M) ==> WF R ==> !x. f x = M x (f % R,x)
%\end{verbatim}
%\end{itemize}

\end{slide}

%\begin{slide}
%\heading{WFREC}
%\medskip
%\begin{itemize}
%\item
%\begin{verbatim}
%WFREC R M = 
%  \x. M x (the_fun (TC R) (\v f. M v (f % R,v)) x % R,x)
%\end{verbatim}
%
%\item $WFREC: (\alpha \rightarrow \alpha \rightarrow bool) \rightarrow
%              ((\alpha \rightarrow \beta) \rightarrow 
%               (\alpha \rightarrow \beta)) \rightarrow
%              \alpha \rightarrow \beta$
%\end{itemize}
%\end{slide}


\begin{slide}
\heading{Relations}
\begin{itemize}
\item Basic wellfounded relations and {\it size}
\begin{eqnarray*}
   WF (\lambda x y. y = Suc\ x) \\
   WF (\lambda L_1 L_2. \exists h. L_2 = Cons\ h\ L_1)
\end{eqnarray*}

\item Derived
\begin{eqnarray*}
  \begin{array}{rcl}
  inv\_image\ R\ f & = & \lambda x y. R\ (f\ x)\ (f\ y) \\
  (R_1\ X\ R_2)\ (s,t)\ (u,v) & = & R_1\ s\ u \lor (s = u) \land R_2\ t\ v \\
  measure & = & inv\_image (<)
  \end{array} \\
  \begin{array}{l}
  \vdash WF\ R \supset WF (inv\_image\ R\ f) \\
  \vdash WF\ R \land WF\ Q \supset WF\ (R\ X\ Q) \\
  \vdash WF\ (<)\   \mbox{hence}\ \vdash \forall f. WF(measure\ f)
  \end{array}
\end{eqnarray*}
\end{itemize}
\end{slide}

\begin{slide}
\heading{Misc. list functions}
\[
\begin{array}{l}
mem\ x\ [] = False\\
mem\ x\ (Cons\ y\ L) = (x=y) \lor mem\ x\ L \\
\\
filter\ P\ [] = [] \\
filter\ P\ (Cons\ x\ L) = \\
{~~~~~}   \mbox{if}\ (P\ x)\ \mbox{then}\ Cons\ x\ (filter\ P\ L) \\
{~~~~~}   \mbox{else}\ filter\ P\ L
\end{array}
\]
\end{slide}

\begin{slide}
\heading{Variant}

\begin{itemize}
\item Input term \\
\begin{eqnarray*}
variant(x,L) = \mbox{if}\ (mem\ x\ L)\ \mbox{then}\ variant(x+1,L)\
\mbox{else}\ x 
\end{eqnarray*}

\item Example behaviours
\[
\begin{array}{c}
 variant(0, [1,2]) = 0 \\
 variant(1, [1,2]) = 3 \\
 variant(1, [1,3]) = 2 \\
\end{array}
\]
\item Termination relations
\[
\begin{array}{l}
measure\  \lambda (x,L). length(filter(\lambda y. x <= y) L) \\
measure\  \lambda (x,L). (Max\ L + 1 ) - x{~~~}   \mbox{(* M. Kaufmann *)}
\end{array}
\]
\end{itemize}
\end{slide}

\begin{slide}
\heading{Extracted Termination Conditions}
\medskip
\[
\begin{array}{l}
 (\forall x\ L. mem\ x\ L \supset \\
{~~~~~~~~}length(filter(\lambda y. x+1 <= y) L)  \\
{~~~~~~~~~~~}          < \\
{~~~~~~~~}        length(filter(\lambda y. x <= y) L)) \\
{~~~~~~}\supset \\
{~~~~~~}variant (x,L) = \\
{~~~~~~~~~~} \mbox{if}\ (mem\ x\ L)\ \mbox{then}\ variant(x+1,L)\
\mbox{else}\ x \mbox{: thm} 
\end{array}
\]
\end{slide}

\begin{slide}
\heading{Variant}
\begin{itemize}
\item Lemma needed for termination proof
\begin{eqnarray*}
 (\forall x. P\ x \supset Q\ x) \supset length(filter\ P\ L) <=
 length(filter\ Q\ L) 
\end{eqnarray*}
\item Main property of variant:
\begin{eqnarray*}
 \neg mem\ (variant (x,L))\ L.
\end{eqnarray*}
\end{itemize}
\end{slide}


\begin{slide}
\heading{Pattern Matching}

The idea is to translate $f(pat_1) = rhs_1 \land \ldots \land f(pat_n) = rhs_n$
to a nested case statement. We use a pattern matching compilation
algorithm based on that of L. Augustsson[85]. 

\begin{itemize}
\item Example - gcd
\[
\begin{array}{l}
 (gcd (0,y)      = y)\\
 (gcd (Suc x, 0) = Suc\  x)\\
 (gcd (Suc\  x, Suc\  y) = \mbox{if}\ (y <= x) \\
{~~~~~~~~~~~~~~~~~~~~~~~~~~~~}\mbox{then}\ gcd(x-y, Suc\  y)  \\
{~~~~~~~~~~~~~~~~~~~~~~~~~~~~}\mbox{else}\ gcd(Suc\  x, y-x)) \\
\end{array}
\]
\item No overlapping patterns allowed.
\end{itemize}
\end{slide}

\begin{slide}
\heading{Case theorems}
\medskip
\[
\begin{array}{l}
(ty\_case\ f_1 ... f_n (C_1 \overline{x}) = f_1\overline{x})  \\
   \ldots \\
(ty\_case\ f_1 ... f_n (C_n\overline{x}) = f_n\overline{x}) \\
\\
case_{prod}\ f\ (p,q) = f\ p\ q
\\
(case_{num}\ v\ f\ 0 = v) \\
(case_{num}\ v\ f\ (Suc\ c) = f\ c) \\
\\
(case_{list}\ v\ f\ [] = v) \\
(case_{list}\ v\ f\ (Cons\ h\ t) = f\ h\ t)
\end{array}
\]

\end{slide}

\begin{slide}
\heading{Generated case expression}
\[
\begin{array}{l}
\lambda gcd\ z.\ case_{prod}\\
{~~~~~} (\lambda v\ v_1.\ case_{nat} v_1  \\
{~~~~~~~~~~~~} (\lambda v_2.\ case_{nat} (Suc\  v_2)   \\
{~~~~~~~~~~~~~~~~~~~~}          (\lambda v_3.\ \mbox{if}\ (v_3 \leq v_2) \\
{~~~~~~~~~~~~~~~~~~~~~~~~~~~~}\mbox{then}\ gcd(v_2-v_3,Suc\  v_3) \\
{~~~~~~~~~~~~~~~~~~~~~~~~~~~~}\mbox{else}\ gcd(Suc\ v_2, v_3-v_2))\ v_1)\ v)\ z.
\end{array}
\]
\end{slide}

\begin{slide}
\heading{Substitute patterns; simplify}
\medskip
\[
\begin{array}{ll} 
 & WF (measure\ \lambda (x,y). x + y) \\
\vdash  &  (gcd(0,y) = y)\ \land \\ 
 & (gcd(Suc\ x,0) = Suc\ x)\ \land \\
 & (gcd(Suc\ x,Suc\ y) = \\
 & {~~~}\mbox{if}\ (y \leq x) \\
 & {~~~}\mbox{then}\ (gcd \!\mid\! (measure\ \lambda (x,y). x + y),
 (Suc\ x,Suc\ y)) (x - y,Suc\ y) \\  
 & {~~~}\mbox{else}\ {~}(gcd \!\mid\! (measure\ \lambda (x,y). x + y),
  (Suc\ x,Suc\ y))(Suc\ x,y - x)).
\end{array}
\]
\end{slide}

\begin{slide}
\heading{Fibonacci}
\[
\begin{array}{l}
\mbox{function}\ \ (<) \\
{~~~~~} Fib\ 0 = 1  \\
{~~~~~} Fib(Suc\  0) = 1 \\
{~~~~~} Fib(Suc(Suc\ x)) = Fib\ x + Fib(Suc\ x)
\end{array}
\]

\end{slide}


\begin{slide}
\heading{Gcd}
\[
\begin{array}{l}
\mbox{function}\ \ (measure (uncurry +)) \\
 gcd (0,y)      = y\\
 gcd (Suc\ x, 0) = Suc\  x\\
 gcd (Suc\  x, Suc\  y) = \mbox{if}\ (y <= x) \\
{~~~~~~~~~~~~~~~~~~~~~~~~~~~~}\mbox{then}\ gcd(x-y, Suc\  y)  \\
{~~~~~~~~~~~~~~~~~~~~~~~~~~~~}\mbox{else}\ gcd(Suc\  x, y-x) \\
\end{array}
\]
\end{slide}

\begin{slide}
\heading{Ackermann}

\[
\begin{array}{l}
\mbox{function}\ \ (pred\ X\ pred) \\
 ack (0,n) = n + 1 \\
 ack (Suc\  m,0) = ack (m,1) \\
 ack (Suc\  m,Suc\  n) = ack (m,ack (Suc\ m,n))
\end{array}
\]

\end{slide}


\begin{slide}
\heading{Quicksort}
\begin{itemize}
\item Definition
\[
\begin{array}{l}
\mbox{function}\ \ (measure(length\circ snd)) \\
{~~~} qsort (ord,[]) = []  \\
{~~~} qsort (ord,Cons\ x\ rst) = \\
{~~~~~~~~~~~~~}    (qsort (ord,filter(\neg \circ ord\ x)\ rst)) \\
{~~~~~~~~~~~~~}    ++ [x] ++ \\
{~~~~~~~~~~~~~}    (qsort (ord,filter(ord\ x)\ rst))
\end{array}
\]


\item Extracted termination conditions
\[
\begin{array}{l}
length(filter\ (ord\ x)\ rst) < length(Cons\ x\ rst) \\
length(filter\ (\neg \circ ord\ x)\ rst) < length(Cons\ x\ rst)
\end{array}
\]
\end{itemize}
\end{slide}


\begin{slide}
\heading{Conditional expressions}

\begin{itemize}
\item Datatype
\begin{verbatim}
cond = A of individual | IF of cond cond cond
\end{verbatim}

\item Definition 
\[
\begin{array}{l}
\mbox{\tt function}\ \ (measure\ M) \\
{~~} norm(A\ i)  =  A\ i \\
{~~} norm(IF(A\  x)\ y\ z)  =  IF (A\  x)\ (norm\ y)\ (norm\ z) \\
{~~} norm(IF(IF\ u\ v\ w)\ y\ z) =  norm(IF\ u\ (IF\ v\ y\ z)\ (IF\
w\ y\ z))
\end{array}
\]

\item Measure function (R. Shostak)
\begin{eqnarray*}
  M (A\ i) & \equiv & 1 \\
  M (IF\ x\ y\ z) & \equiv & M x + (M x \ast M y) + (M x \ast M z)
\end{eqnarray*}

\end{itemize}
\end{slide}

\begin{slide}
\heading{Conditional expressions - contd.}
\noindent Expanded termination conditions

\[
\begin{array}{l}
   M z < 1 + M y + M z\ \land \\
   M y < 1 + M y + M z\ \land \\
   M u +  M u * (M v + M v * M y + M v * M z)\ + \\
   M u * (M w + M w * M y + M w * M z) \\
  < \\
  (M u + M u * M v + M u * M w)\ + \\
  (M u + M u * M v + M u * M w) * M y\ + \\
  (M u + M u * M v + M u * M w) * M z
\end{array}
\]

\end{slide}

\begin{slide}
\heading{Nested Recursion}

\begin{itemize}
\item Definition 
\[
\begin{array}{l}
\mbox{\tt function}\ \ (measure (\lambda x. 101 - x)) \\
 Ninety1\ x = \mbox{if}\; x>100\; \mbox{then}\;x-10\; \mbox{else}\;
 Ninety1(Ninety1(x+11)) 
\end{array}
\]

\item Termination conditions
\[
\begin{array}{l}
\neg x > 100 \supset 101 - (x + 11) < 101 - x \\
\neg x > 100 \supset 
\begin{array}[t]{l}
     (101 - (Ninety1 \!\mid\! (\lambda x y. 101 - x < 101 - y),x) (x + 11) \\
\hspace{-3mm}  <  101 - x)
\end{array}
\end{array}
\]

The second is equivalent to

\[ x\leq 100 \supset x < Ninety1(x+11). \]

\end{itemize}
\end{slide}

\begin{slide}
\heading{Custom induction theorems}
\begin{itemize}
\item Gcd
\[
\begin{array}{l}
\forall P.\ (\forall y. P (0,y))\ \land \\
{~~~~~~~}    (\forall x. P (Suc\ x,0))\ \land \\
{~~~~~~~}    (\forall y x. (\neg y \leq x \supset P(Suc\  x,y - x))\ \land \\
{~~~~~~~~~~~~~~~} (y \leq x \supset P(x - y,Suc\ y))\\
{~~~~~~~~~~~~~~~}\supset P (Suc\  x,Suc\ y)) \\
{~~~~~~~}\supset \forall v\ v_1. P (v,v_1)
\end{array}
\]

\item Quicksort
\[
\begin{array}{l}
[TC_1, TC_2] \\
\vdash \forall P.\ (\forall ord. P (ord,[]))\ \land \\
{~~~~~~~~~}(\forall ord\ x\ rst. P (ord,filter (ord\ x) rst)\ \land \\
{~~~~~~~~~~~~~~~~~~~~~~~~}P (ord,filter (\neg \circ ord\ x) rst) \\
{~~~~~~~~~~~~~~~~~~~~~~~~}\supset P(ord,CONS\ x\ rst)) \\
{~~~~~~~~~}\supset \forall v\ v_1. P (v,v_1)
\end{array}
\]
\end{itemize}
\end{slide}


\begin{slide}
\heading{Future work}

\begin{itemize}
\item Compute R
\item Decrypt nestedness
\item Larger scale examples
\end{itemize}
\end{slide}

\end{document}
