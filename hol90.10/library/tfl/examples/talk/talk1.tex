%\documentstyle[fancybox,sem-a4,itmath]{seminar}
\documentstyle[fancybox,sem-a4]{seminar}
\input{amssym.def}
\input{amssym.tex}
\newcommand{\imp}{\supset}
\newcommand{\itelse}[3]{\mbox{$\mbox{if}\ {#1}\ \mbox{then}\ {#2}\
    \mbox{else}\ {#3}$}}
\newcommand{\constr}[1]{\mbox{\sf {#1}}}
\newcommand{\transl}[1]{\mbox{$\{ \!\! | {#1} | \!\! \}$}}

\landscapeonly 
%\portraitonly

% Personal hints concerning printing:
% 1. enable \landscapeonly and print by 
%    dvips -t landscape slides
% 2. enable \portraitonly and use portrait-option to print the rest by
%    dvips slides 

%\slidesmag{2}

\newcommand{\heading}[1]{%
  \begin{center}
    \large\bf
    \shadowbox{\makebox{#1}}%
  \end{center}
  \vspace{1ex minus 1ex}}

\newcommand{\Infer}[2]%
{\frac{\vbox{\hbox{$\displaystyle #1$}\hbox{\vspace{.1ex}}}}%
      {\vbox{\hbox{\vspace{.3ex}}\hbox{$\displaystyle #2$}}}}


\begin{document}


%\begin{slide}
%\heading{More Functions in Higher Order Logic}
%\bigskip
%\bigskip
%\begin{center}
%  Konrad Slind  \\
%TU M{\"u}nchen, Cambridge U.
%\end{center}
%\end{slide} 
%
%\begin{slide}
%\heading{Plan}
%
%\begin{enumerate}
%\item Introduction
%\item Logical Basis
%\item Design
%\item Custom Induction Theorems
%\item Examples
%\end{enumerate}
%\end{slide} 
%
%
%\begin{slide}
%\heading{Variant}
%\[\begin{array}{l}
% (\forall x\ L. \ mem\ x\ L \imp \\
%{~~~~~~~~}length(filter(\lambda y. x+1 \leq y) L) < length(filter(\lambda
%y. x \leq y) L)) \\ 
%\supset \\
%variant (x,L) =  \mbox{if}\ (mem\ x\ L)\ \mbox{then}\ variant(x+1,L)\ \mbox{else}\ x \mbox{: thm} 
%\end{array}
%\]
%
%\noindent Lemma needed for termination proof:
%\begin{eqnarray*}
% (\forall x.\ P\ x \supset Q\ x) \supset length(filter\ P\ L) \leq
% length(filter\ Q\ L) 
%\end{eqnarray*}
%
%\end{slide}
%
%\begin{slide}
%\heading{Variant}
%\begin{itemize}
%\item Induction theorem
%\begin{eqnarray*}
% \vdash \forall P.\;(\forall x\; L. (mem\;x\;L \imp P(Suc\;x,L)) \imp
%P (x,L)) \imp \forall v\ v_1.\ P (v,v_1)
%\end{eqnarray*}
%\item Main property of variant:
%\begin{eqnarray*}
% \neg mem\ (variant (x,L))\ L.
%\end{eqnarray*}
%\end{itemize}
%\end{slide}
%
%\begin{slide}
%\heading{Naive program tactic}
%\begin{itemize}
%\item Apply induction theorem
%\item Break into cases (from clauses in def'n.)
%\item Unroll recursive function once
%\item Do all case splits (from control structures in program)
%\end{itemize}
%\end{slide}
%
%\begin{slide}
%\heading{Variant correctness}
%\begin{itemize}
%\item
%\[\begin{array}{rl}
% Goal: & \neg(mem\ x\ L) \\ \hline
% Assums: & mem\ x\ L \imp \neg(mem\ (variant\ (\constr{Suc}\ x,L))\ L) \\
%         &    \neg(mem\ x\ L)
%\end{array}
%\]
% 
%\item
%\[\begin{array}{rl}
% Goal: & \neg(mem\ (variant(\constr{Suc}\ x,L))\ L) \\ \hline
% Assums: & mem\ x\ L \imp \neg(mem\ (variant (\constr{Suc}\ x,L))\ L) \\
%         & mem\ x\ L
%\end{array}
%\]
%\end{itemize}
%\end{slide}
%
%
%\begin{slide}
%\heading{Custom induction theorems}
%From a function definition
%\begin{eqnarray*}
%  f (pat_1) & = & rhs_1[f(a_{11}),\ldots,f(a_{1{k_1}})] \\
%  & \vdots & \\
%  f (pat_n) & = & rhs_n[f(a_{n1}),\ldots,f(a_{n{k_n}})] 
%\end{eqnarray*}
%\noindent the extraction process gives us the termination conditions
%\[\begin{array}{lcl}
%  WF(R), & & \\
%  \forall(\Gamma(a_{11}) \imp R\;a_{11}\;pat_1), & \ldots, &
%  \forall(\Gamma(a_{1{k_1}}) \imp R\;a_{1{k_1}}\;pat_1), \\
%  & \vdots & \\
%  \forall(\Gamma(a_{n1}) \imp R\;a_{n1}\;pat_n),& \ldots, &
%  \forall(\Gamma(a_{n{k_n}}) \imp R\;a_{n{k_n}}\;pat_n)
%\end{array}\]
%\end{slide}
%
%\begin{slide}
%\heading{Target theorem}
%\[\begin{array}{ll}
%    \left( \forall 
%                 \left(\begin{array}{l}
%                        \begin{array}[b]{ll}
%                          (\forall(\Gamma(a_{11}) \imp P\;a_{11})) & \land  \\
%                          \vdots & \land \\
%                          (\forall(\Gamma(a_{1{k_1}}) \imp P\;a_{1{k_1}})) &
%                         \end{array}
%                        \end{array}
%                 \right) \imp P(pat_1)\right ) & \land \\
%   & \hspace{0.1cm}\vdots \\
%   & \land \\
%   \left ( \forall
%       \left( \begin{array}{l}
%                 \begin{array}[t]{ll}
%                 (\forall(\Gamma(a_{n1}) \imp P\;a_{n1})) & \land \\
%                 \vdots & \land \\
%                 (\forall(\Gamma(a_{n{k_n}}) \imp P\;a_{n{k_n}})) &
%                 \end{array}
%               \end{array}
%       \right) \imp P(pat_n)\right ) & \imp \forall v.\ P\;v. 
% \end{array}\]
%\end{slide}
%
%\begin{slide}
%\heading{Pattern completeness}
%
%\[
% \forall x.\ (\exists \overline{y}.\ x = pat_1(\overline{y})) \lor
%\ldots \lor (\overline{y}.\ x = pat_n(\overline{y})) \]
%
%\noindent Example.
%
%\[\begin{array}{l}
% \forall x.\ (x = (0,0)) \lor (\exists y.\ x = (0,\constr{Suc}\;y)) \lor \\
%{~~~~~} (\exists y.\ x = (\constr{Suc}\;y,0)) \lor (\exists y\;z.\ x = (\constr{Suc}\;y, \constr{Suc}\;z))
%\end{array} \]
%
%\noindent Implemented by adaptation of Augustsson's pattern matching
%algorithm to perform explicit proof.
%\end{slide}
%
%\begin{slide}
%\heading{Permutations}
%\[ perm\; l_1\; l_2 \equiv \forall x.\ filter\ (op = x)\ l_1 = filter\
%(op = x)\ l_2\] 
%
%\noindent $perm$ is an equivalence relation and
%
%\[perm\; l_1\; l_3 \land perm\;l_2\;l_4 \imp perm (l_1 @
%l_2) (l_3 @ l_4)\]
% 
%\end{slide}
%
%% \begin{slide}
%% \heading{Sorting}
%% \[\begin{array}{lr}
%%   tt\_relation\ R \equiv & \\
%% {~~~~~~}  (\forall x\;y\;z.\ R\;x\;y \land R\;y\;z \imp R\;x\;z)\ \land
%% & (* Transitive *) \\
%% {~~~~~~}  (\forall x\;y.\ R\;x\;y \lor R\;y\;x) &  (* Total *)
%% \end{array}
%% \]
%
%
% \begin{slide}
% \heading{Sorting}
% \[\begin{array}{ll}
%   tt\_relation\ R \equiv & (\forall x\;y\;z.\ R\;x\;y \land R\;y\;z
% \imp R\;x\;z)\ \land {~~~~~~~~~~~~~~} \\
% &  (\forall x\;y.\ R\;x\;y \lor R\;y\;x) \end{array}
% \]
%\[\begin{array}{l}
%(finiteRchain (R, []) = True)\ \land \\
%(finiteRchain (R, [x]) = True)\ \land \\
%(finiteRchain (R, x::y::rst) = R\;x\;y \land finiteRchain(R, y::rst))
%\end{array}
%\]
%
%\[\begin{array}{l}
%performs\_sorting\ f \equiv \\
%{~~~~~}\forall R.\ tt\_relation\ R \imp \forall L.\ perm\;L\ (f(R,L))\
%\land {~~~~~~~~~~~~~~~~~~~} \\
%{~~~~~~~~~~~~~~~~~~~~~~~~~~~~~~~~~~~~~}finiteRchain(R,\ f(R,L))
%\end{array}
%\]
%
%\end{slide}
%
%\begin{slide}
%\heading{Naive Quicksort}
%\[\begin{array}{l}
%(qsort(R,[\;]) = [\;])\ \land \\
%(qsort(R,h::t) = \\
%{~~~~~}qsort(R, filter\ (\neg \circ R\; h)\ t)@[h]@qsort(R,filter\ (R\; h)\ t)).
%\end{array}
%\]
%\noindent Induction.
%
%\[\begin{array}{rl}
% \forall P. & (\forall R.\ P (R,[])) \ \land \\
%    & (\forall R\;x\;rst.\ P (R,filter (R\; x)\ rst) \ \land \\
%    & {~~~~~~~~~~~~~~} P (R,filter (\neg \circ R\; x)\ rst)\  \imp P
%    (R,x::rst))  \\
% &  \imp \forall v\;v_1.\ P (v,v_1).
%\end{array}\]
%\end{slide}
%
%\begin{slide}
%\heading{$\forall R\ L.\ perm\ L\ (qsort(R,L))$}
%
%$\begin{array}{rl} 
%Goal: & perm\ [\;] \ [\;] \\
%      &               \\
%Goal: & perm{~} (x :: rst) \\
%      & {~~~~~~~~~} (qsort (R,filter\ (\neg \circ R\ x)\ rst)\\
%      & {~~~~~~~~~}  @ [x] @ \\
%      & {~~~~~~~~~}  qsort(R,filter\;(R x)\; rst)) \\ \hline 
%Assums: & perm (filter\ (R\ x)\ rst)\ (qsort (R,filter\ (R\ x)\ rst)) \\
%        & perm (filter\ (\neg \circ  R\ x)\ rst)\  (qsort (R,filter\ (\neg \circ
%           R\ x) rst)) 
%\end{array}
%$
%\end{slide}
%
%
%\begin{slide}
%\heading{Faster Quicksort}
%
%\[\begin{array}{l}
%{~~}(part(P,[\;],l_1,l_2) = (l_1,l_2))\ \land \\
%{~~}(part(P,h::t,l_1,l_2) = \\
%{~~~~~~~~~}\mbox{if}\ P\ h\ \mbox{then}\ part(P,t,h::l_1,l_2) {~~~~~~~~~~~~~~~~~~~~~~~~~}\\
%{~~~~~~~~~~~~~~~~~~} \mbox{else}\ \; part(P,t,l_1,h::l_2))
%\end{array}
%\]
%
%\[
%\begin{array}{l}
%{~~}(qsort(R,[\;]) = [\;])\ \land \\
%{~~}(qsort(R,h::t) = \\
%{~~~~~~~~}\mbox{let}\ (l_1,l_2) = part(\lambda y.\ R\ y\ h,
%t, [\;],[\;]){~~~~~~~~~~~~~~~~~~~~~~~~}\\ 
%{~~~~~~~~}\mbox{in} \\
%{~~~~~~~~}qsort(R,l_1)@[h]@qsort(R,l_2)).
%\end{array}
%\]
%\end{slide}
%
%\begin{slide}
%\heading{Faster Quicksort induction}
%\[
%\begin{array}{l}
%\forall P. \\
%{~~~~} (\forall R.\ P(R,[\;]))\ \land \\
%{~~~~}(\forall R\;h\;t. \\
%{~~~~~~~~~~}(\forall l_1 l_2.\ ((l_1,l_2) = part((\lambda y.\
%R\;y\;h),t,[\;],[\;])) \imp P (R,l_2))\ \land \\ 
%{~~~~~~~~~~}(\forall l_1 l_2.\ ((l_1,l_2) = part ((\lambda y.\
%R\;y\;h),t,[\;],[\;])) \imp P(R,l_1)) \\
%{~~~~~~~~}  \imp P (R,h::t)) \\
%{~~}\imp \forall v\;v_1.\ P (v,v_1)
%\end{array}
%\]
%\end{slide}
%
%\begin{slide}
%\heading{$\forall R\ L.\ perm\ L\ (fqsort(R,L))$}
%
%$\begin{array}{rl} 
%Goal: & perm\ [\;] \ [\;] \\
%      &               \\
%Goal: & perm{~} (x :: rst) \\
%      & {~~~~~~~~~} (\mbox{let}\ (l_1,l_2) = part((\lambda
%y. R\;y\;x),rst,[\;],[\;]) \\ 
%      & {~~~~~~~~~~} \mbox{in} \\
%      & {~~~~~~~~~~} fqsort (R,l_1) @ [x] @ fqsort (R,l_2)) \\ \hline 
%Assums: & \forall l_1 l_2.\ ((l_1,l_2) = part((\lambda y. R\;y\;x),rst,[\;],[\;])) \\
%      & {~~~~~~~~~~} \imp  perm\ l_1\ (fqsort (R,l_1)) \\
%      & \forall l_1 l_2.\ ((l_1,l_2) = part((\lambda y. R\;y\;x),rst,[\;],[\;])) \\ 
%      & {~~~~~~~~~~} \imp perm\; l_2\ (fqsort (R,l_2))
%
%\end{array}
%$
%\end{slide}
%
%\begin{slide}
%\heading{Let expressions}
%\[\begin{array}{rcl}
% LET\;f\;x & \equiv & f\;x \\
%pair\_case\;f\;(x,y) & \equiv & f\;x\;y
%\end{array}\]
%\noindent Parse and print translations.
%\[\begin{array}{l}
%\transl{\mbox{let}\; x = M\ \mbox{in}\ N} = LET\; (\lambda
%x. \transl{N})\; \transl{M} \\
%\transl{\mbox{let}\;(x,y) = M\ \mbox{in}\ N} = LET\; (pair\_case
%(\lambda x\;y. \transl{N}))\; \transl{M} 
%\end{array}
%\]
%
%\noindent Introduction rule(s).
%\[\begin{array}{rcl}
% (\forall x.\ (x=M) \imp P(N\;x)) & \imp & P(\mbox{let}\ x = M\ \mbox{in}\ N\;x) \\
% (\forall x\;y.\ ((x,y)=M) \imp P(N\;x\;y)) & \imp & P(\mbox{let}\ (x,y) = M\ \mbox{in}\ N\;x\;y)
%\end{array}\]
%
%\end{slide}
%
% \begin{slide}
% \heading{Next step}
% 
% $\begin{array}{rl} 
% Goal: & perm{~} (x :: rst) \\
%       & {~~~~~~~~~} (fqsort (R,l_1) @ [x] @ fqsort (R,l_2)) \\ \hline 
% Assums: & \forall l_1 l_2.\ ((l_1,l_2) = part((\lambda y. R\;y\;x),rst,[\;],[\;])) \\
%       & {~~~~~~~~~~} \imp  perm\ l_1\ (fqsort (R,l_1)) \\
%       & \forall l_1 l_2.\ ((l_1,l_2) = part((\lambda y. R\;y\;x),rst,[\;],[\;])) \\ 
%       & {~~~~~~~~~~} \imp perm\; l_2\ (fqsort (R,l_2)) \\
%       & (l_1,l_2) = part((\lambda y.\ R\;y\;x),rst,[\;],[\;])
% \end{array}
% $
% 
% \noindent Lemma: 
% \[ ((a_1,a_2) = part(P,L,l_1,l_2)) \imp perm (L@l_1@l_2)\ (a_1@a_2)\]
% \end{slide}
% 
% \begin{slide}
% \heading{Dutch National Flag}
% 
% \verb+datatype colour = R | W | B+
% 
% \[\begin{array}{lclr}
%   (Dnf []          & = &  ([], False)) & \land \\
%   (Dnf (W::R::rst) & = &  (R::W::rst, True)) & \land \\ 
%   (Dnf (B::R::rst) & = &  (R::B::rst, True)) & \land \\ 
%   (Dnf (B::W::rst) & = &  (W::B::rst, True)) & \land \\ 
%   (Dnf (x::rst)    & = &  (cons\;x\sharp I) (Dnf\; rst)) &
% \end{array}
% \]
% 
% \[ (f\sharp g) (x,y) \equiv (f\;x, \ g\; y) \]
% 
% \end{slide}
% 
% \begin{slide}
% \heading{Expanded definition}
% \[\begin{array}{lclr}
%     (Dnf []  & = &  ([],  False)) & \land \\
%     (Dnf [B] & = &  ([B], False)) & \land \\
%     (Dnf [W] & = &  ([W], False)) & \land \\
%     (Dnf (W::R::rst) & = &  (R::W::rst, True)) & \land \\
%     (Dnf (B::R::rst) & = &  (R::B::rst, True)) & \land \\
%     (Dnf (B::W::rst) & = &  (W::B::rst, True)) & \land \\
%     (Dnf (B::B::rst)  & = &  (cons\;B \sharp I) (Dnf (B::rst))) & \land \\
%     (Dnf (W::B::rst)  & = &  (cons\;W \sharp I) (Dnf (B::rst))) & \land \\
%     (Dnf (W::W::rst)  & = &  (cons\;W \sharp I) (Dnf (W::rst))) & \land \\
%     (Dnf (R::rst)    & = &  (cons\;R \sharp I) (Dnf\; rst)) &
% \end{array}
% \]
% \end{slide}
% 
% \begin{slide}
% \heading{Induction}
% \[\begin{array}{rl}
% \forall P.\ & P [] \ \land P [B] \ \land  P [W] \ \land \\
%  &        (\forall rst.\  P (W::R::rst)) \ \land \\
%  &        (\forall rst.\  P (B::R::rst)) \ \land \\
%  &        (\forall rst.\  P (B::W::rst)) \ \land \\
%  &        (\forall rst.\  P (B::rst) \imp P (B::B::rst)) \ \land \\
%  &        (\forall rst.\  P (B::rst) \imp P (W::B::rst)) \ \land \\
%  &        (\forall rst.\  P (W::rst) \imp P (W::W::rst)) \ \land \\
%  &        (\forall rst.\  P\; rst \imp P (R::rst)) \\
% \imp  & \forall v.\ P\; v.
% \end{array}
% \]
% \end{slide}
% 
% \begin{slide}
% \heading{Control}
% \[\begin{array}{l}
% flag\ L = \begin{array}[t]{l}\mbox{let}\ (l_1, delta) = Dnf\ L \\
%                  \mbox{in if}\ delta\ \mbox{then}\ flag\; l_1 \ 
%                  \mbox{else}\ l_1
%           \end{array}
% \end{array}
% \]
% \noindent Induction.
% 
% \[\begin{array}{l}
%  \forall P.\ (\forall L.\ (\forall l_1\ delta.\ ((l_1,delta) = Dnf\;L) \land delta \imp P\;l_1) \imp P\;L) \\
% {~~~~} \imp \forall v.\ P\;v 
% \end{array}\]
% \end{slide}
% 
% \begin{slide}
% \heading{Correctness}
% \[\begin{array}{l}
% \forall L.\ \exists l_1 l_2 l_3.\ (flag\;L = l_1@l_2@l_3)  \land \\
% {~~~~~~~~~~}(\forall x.\ mem\;x\;l_1 \imp (x=R))\ \land \\
% {~~~~~~~~~~}(\forall x.\ mem\;x\;l_2 \imp (x=W))\ \land \\
% {~~~~~~~~~~}(\forall x.\ mem\;x\;l_3 \imp (x=B)) \\
% = \\
% \forall L.\ flag\;L = filter (op = R)\ L\ @\ \\
% {~~~~~~~~~~~~~~~~~~}filter (op = W)\ L\ @\ \\
% {~~~~~~~~~~~~~~~~~~} filter(op= B)\ L
% \end{array} \]
% 
% \end{slide}
% 
% \begin{slide}
% \[\begin{array}{rl}
% Goal: & (\mbox{let}\ (l_1,delta) = Dnf\ L\ \mbox{in\ if}\ delta\
% \mbox{then}\ flag\;l_1\ \mbox{else}\ l_1) \\ 
%       & = filter (op = R)\ L @ filter (op = W)\ L @ filter (op = B)\ L
% \\ \hline
% Assums: & \forall l_1\; delta.\ ((l_1,delta) = Dnf\;L) \land delta \\
%         & {~~~~~~~~~~~~} \imp (flag\;l_1 = filter (op = R)\ l_1 @ \\
%         & {~~~~~~~~~~~~~~~~~~~~~~~~~~~~~~}filter (op = W)\ l_1 @ \\
%         & {~~~~~~~~~~~~~~~~~~~~~~~~~~~~~~}filter (op = B)\ l_1) \\
% \end{array}
% \]
% 
% \noindent Now lift let, case split, and use IH (in positive case).
% \end{slide}
% 
% \begin{slide}
% \[\begin{array}{rl}
% Goal\ 1: & filter(op = R)\ l_1 @ filter (op = W)\ l_1 @ filter (op = B)\ l_1 \\
%          & = filter (op = R)\ L @ filter (op = W)\ L @ filter (op = B)\ L \\
% \hline
% Assums: & (l_1,True) = Dnf\; L \\
%  & \{ lemma:\   \forall L.\ perm\ L\ (fst (Dnf\;L)) \} \\
%  & \\
%  & \\
% Goal\ 2: & l_1 = filter (op = R) L @ filter (op = W) L @ filter (op = B) L \\
% \hline
% Assums: & (l_1,False) = Dnf\; L \\
%  & \{ lemmas: \forall L\ l_1.\ ((l_1,False) = Dnf\;L) \imp (l_1=L) \\
%  & {~~~~~~}\forall L.\ ((L,False) = Dnf\ L) \imp (L = filter (op = R) L @ \\
%  &  {~~~~~~~~~~~~~~~~~~~~~~~~~~~~~~~~~~~~~~~~~~~~~~~~~~~}filter (op = W) L @  \\
%  &  {~~~~~~~~~~~~~~~~~~~~~~~~~~~~~~~~~~~~~~~~~~~~~~~~~~~~}filter (op = B) L) \}
% \end{array}
% \]
% \end{slide}  
% 
% \begin{slide}
% \heading{Ackermann}
% \[
% \begin{array}{l}
% {~~~~~} (ack (0,n)  =  n+1)\ \land \\
% {~~~~~} (ack (\mbox{\sf Suc}\ m,0)  =  ack (m, 1))\ \land \\
% {~~~~~} (ack (\mbox{\sf Suc}\ m, \mbox{\sf Suc}\ n) =  ack (m, ack
% (\mbox{\sf Suc}\ m, n))) 
% \end{array}
% \]
% 
% \noindent Induction theorem
% 
% \[
% \begin{array}{l}
% \forall P.\ (\forall n.\ P (0,n)) \land \\
% {~~~~~} (\forall m.\ P (m,1) \imp P (\constr{Suc}\ m,0)) \land \\
% {~~~~~} (\forall m\;n.\ P (m,ack (\constr{Suc}\;m,n)) \land P(\constr{Suc}\; m,n) \imp P
% (\constr{Suc}\; m,\constr{Suc}\; n)) \\
% {~~~~} \imp \forall v\;v_1.\ P (v,v_1)
% \end{array}
% \]
% \end{slide}
% 
% \begin{slide}
% \heading{91}
% 
% \[
% \begin{array}{l}
% ninety1\ x = \mbox{if}\ (x>100)\ \mbox{then}\ x-10 \ 
%         \mbox{else}\ ninety1(ninety1(x+11)).
% \end{array}
% \]
% \noindent Termination measure =  $\lambda x.\ 101 - x$
% 
% \noindent Induction theorem
% \[\begin{array}{l}
% \forall P. \\
% {~~~} (\forall x.\ (\neg(x > 100) \imp P (x + 11) \land P (ninety1 (x +
% 11))) \imp P\ x) \\
% \imp \forall v.\ P\ v
% \end{array}\]
% 
% \noindent But first need to prove
% \[\forall x. \neg(x>100) \imp x < ninety1(x+11) \]
% ... by induction on the termination measure!
% \end{slide}
% 
% \begin{slide}
% \heading{Correctness}
% \[\forall n.\ ninety1\ n = \itelse{n>100}{n-10}{91} \]
% 
% \[\begin{array}{rl}
% Goal: & ninety1 (ninety1 (x + 11)) = 91 \\ \hline
% Assums: \! & \! \neg(x>100) \imp \\
%         & {~~~~}(ninety1(x+11) = \itelse{x + 11 > 100}{x+1}{91})  \land \\
%         & {~~~~}(ninety1 (ninety1 (x + 11)) = \\
%         & {~~~~~~~~~~}\mbox{if}\ {ninety1 (x + 11) > 100} \\
%         & {~~~~~~~~~~~~}\mbox{then}\ {ninety1 (x + 11) - 10}\
%         \mbox{else}\ {91}) \\ 
%         & \neg(x > 100)
% \end{array}
% \]
%\end{slide} 

 \begin{slide}
 \heading{Unification}
 
 \[
 \begin{array}{rl}
\mbox{\tt datatype}\ \alpha\ uterm = & Var\; (\alpha)  \\
                  | & Const\ (\alpha)    \\
                  | & Comb\ (\alpha\ uterm)\ (\alpha\ uterm) \\
 & \\
\mbox{\tt datatype}\ \alpha\ result = & Fail  \\
                  | & Subst\ (\alpha\ list)
 \end{array}
 \]
 
\[
\begin{array}{l}
Unify(Const\;m, Const\;n)  =  if\;(m = n)\;then\;Subst[]\;else\;Fail \\
Unify(Const\;m, Comb\;M\;N)  =   Fail \\
Unify(Const\;m, Var\;v)      =   Subst[(v,Const\;m)] \\
Unify(Var\;v, M)  =  \itelse{(Var\;v <:M)}{Fail}{Subst[(v,M)]} \\
Unify(Comb\;M\;N, Const\;x)  =   Fail \\
Unify(Comb\;M\;N, Var\;v)  = \itelse{(Var\;v <:Comb\;M\;N)}{Fail}{Subst[(v,Comb\;M\;N)]}  \\
Unify(Comb\;M_1\;N_1, Comb\;M_2\;N_2)  =   \\
{~~~~~~~} case\;Unify(M_1,M_2) \\
{~~~~~~~~~} of\ Fail => Fail  \\
{~~~~~~~~~~~} | \ Subst\ \theta  => case\;Unify(N_1 \lhd \theta, N_2 \lhd
\theta) \\ 
{~~~~~~~~~~~~~~~~~~~~~~~~~~~~~~} of\ Fail => Fail \\
{~~~~~~~~~~~~~~~~~~~~~~~~~~~~~~~~}  | \ Subst\ \sigma  => Subst(\theta \bullet \sigma)
\end{array}
\]
 
\end{slide}

\begin{slide}

\[
\begin{array}{rl}
\forall \phi.\ & (\forall m\;n.\ \phi  (Const\;m, Const\;n))  \ \land \\
& (\forall m\;M\;N.\ \phi  (Const\;m, Comb\;M\;N))  \ \land \\
& (\forall m\;v.\ \phi  (Const\;m, Var\;v)) \ \land \\
& (\forall v\;M.\ \phi  (Var\;v, M)) \ \land \\
& (\forall M\;N\;x.\ \phi  (Comb\;M\;N, Const\;x)) \ \land \\
& (\forall M\;N\;v.\ \phi  (Comb\;M\;N, Var\;v)) \ \land \\
& (\forall M_1\;N_1\;M_2\;N_2. \\
& {~~~}(\forall  \theta.\ Unify (M_1, M_2) = Subst\ \theta  \imp \phi
(N_1 \lhd  \theta, N_2 \lhd \theta)) \\ 
& {~~~}\land\phi  (M_1, M_2)  \\
& {~~~}\imp\phi  (Comb\;M_1\;N_1, Comb\;M_2\;N_2)) \\
&  \imp \forall v\;v_1.\ \phi (v, v_1)
\end{array}
\]
\end{slide}

\begin{slide}
\heading{Unify -- Correctness}
\[\forall \theta.\ Unify(P,Q) = Subst\  \theta  \imp MGUnifier\ \theta\;P\;Q \]
\[ \forall \theta. Unify (P,Q) = Subst\  \theta  \imp Idem\ \theta \]
\end{slide}

\begin{slide}
\heading{Underspecification}
\[\begin{array}{l}
nth(0, h::t) = h \\
nth(\constr{Suc}\ n, h::t) = nth(n,t)
\end{array}\]

\noindent Induction.
\[
\begin{array}{rl}
\forall \phi.\ & \phi (0,[\;])\ \land \\
& (\forall n.\ \phi  (\constr{Suc}\ n,[\;]))  \ \land \\
& (\forall h\;t.\ \phi  (0,h :: t)) \ \land \\
& (\forall h\;n\;t.\ \phi(n,t) \imp \phi (\constr{Suc}\ n,\ h::t)) \\
&  \imp \forall v\;v_1.\ \phi (v, v_1)
\end{array}
\]
\end{slide}
\begin{slide}
\heading{Nth -- a property}
\[\begin{array}{l}
funpow(0, f, x) = x \\
funpow(\constr{Suc}\ n, f, x) = f(funpow(n,f,x))
\end{array}\]

\noindent Property.
\[\forall n\ L.\ n < length\ L \imp (nth(n,L) = hd(funpow(n,tl,L))). \]

\noindent Lemma. $funpow (n, f, f\ L) = f (funpow (n, f, L))$ 

\end{slide}

\begin{slide}
\heading{Higher-order recursion}
\[\begin{array}{l}
    Occ (v_1, Var\ v_2) = (v_1 = v_2) \\
    Occ (v, Const\ x) = False \\
    Occ (v, App\ M\ N) = exists\ (\lambda x.\ Occ(v,x))\ [M, N]
\end{array}\]

\noindent Need the context rule
\[\begin{array}{l}
   (L = L_1) \land (\forall y.\ mem\ y\ L_1 \imp (P\ y = P_1\ y)) \\ 
{~~~}   \imp exists\ P\ L = exists\ P_1\ L_1
\end{array}\]

\[\begin{array}{l}
 [WF\ R, \forall x\ M\ N\ v.\ mem\ x\ [M,N] \imp R (v,x)\ (v, App\ M\ N)] \\
\vdash \\
 ((Occs (v_1,Var\ v_2) = v_1 = v_2)\ \land \\
  (Occs (v,Const\ x) = False)\ \land \\
  (Occs (v,App\ M\ N) = exists\ (\lambda x.\ Occs (v,x))\ [M, N])) \\
{~~~}  \land \\
 \forall \phi.\ (\forall v_1\; v_2.\ \phi (v_1,Var\ v_2))\ \land  \\
{~~~~~~}       (\forall v\ x.\ \phi (v,Const\ x))\ \land \\
{~~~~~~}       (\forall v\ M\ N.\ (\forall x.\ mem\ x\ [M, N] \imp \phi (v,x))
 \imp \phi (v,\ App\ M\ N)) \\
{~~~~~~}       \imp \\
{~~~~~~}\forall v\ v_1.\ \phi (v,v_1)
\end{array}\]

\end{slide}

\end{document}
